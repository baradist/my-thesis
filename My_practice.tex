%%%% Шаблон Отчета по практике <<SPbPU-student-thesis-template>>  %%%%
%%
%%   Создан на основе глубокой переработки шаблона российских кандидатских и докторских диссертаций [1]. 
%%   
%%   Полный список различий может быть получен командами git.
%%   Лист авторов-составителей расположен в README.md файле.
%%   Подробные инструкции по использованию в [1,2].
%%   
%%   Рекомендуем установить TeX Live + TeXstudio
%%   <<Стандартная>> компиляция 2-3 РАЗА с помощью pdflatex + biber (для библиографии)     
%%  
%%%% Student thesis template <<SPbPU-student-thesis-template>> %%%%
%%
%%   Created on the basis of deep modifification of the Russian candidate and doctorate thesis template [1]. 
%%   
%%   Full list of differences can be achieved by git commands.
%%   List of template authors can be seen in the README.md file.
%%   Detailed instructions of usage, see, please in [1,2].
%%     
%%   [1] github.com/AndreyAkinshin/Russian-Phd-LaTeX-Dissertation-Template 
%%   [2] Author_guide_SPBPU-student-thesis-template.pdf
%%   
%%   It is recommended to install TeX Live + TeXstudio   
%%   Default compilation 2-3 TIMES with pdflatex + biber (for the bibliography)
%%  
\input{template_settings/ch_preamble} % лучше не редактировать / please, keep unmodified

\setcounter{docType}{2} % лучше не редактировать / please, keep unmodified

%%%% Настройки автора / Author settings
%% 
\input{my_folder/my_settings} % добавляем свои команды / update your commands


\begin{document} % начало документа
	


%%% Внесите свои данные - Input your data
%%
%%
\newcommand{\Author}{О.Ю.\,Григорьев} % И.О. Фамилия автора
\newcommand{\AuthorFull}{Григорьев Олег Юрьевич} % Фамилия Имя Отчество автора
\newcommand{\AuthorFullDat}{Григорьеву Олегу Юрьевичу} % Фамилия Имя Отчество автора в дательном падеже (Кому? Студенту...)
\newcommand{\AuthorFullVin}{Григорьева Олега Юрьевича} % в винительном падеже (Кого? что?  Програмиста ...)
\newcommand{\AuthorPhone}{+7-911-088-88-48} % номер телефорна автора для оперативной связи
\newcommand{\Supervisor}{В.Г.\,Пак} % И. О. Фамилия научного руководителя
\newcommand{\SupervisorFull}{Пак Вадим Геннадьевич} % Фамилия Имя Отчество научного руководителя
\newcommand{\SupervisorVin}{В.Г.\,Пака} % И. О. Фамилия научного руководителя  в винительном падеже (Кого? что? Руководителя ...)
\newcommand{\SupervisorJob}{доцент ВШИСиСТ} %
\newcommand{\SupervisorJobVin}{доцента ВШиСТ} % в винительном падеже (Кого? что?  Програмиста ...)
\newcommand{\SupervisorDegree}{к.ф.-м.н.} %
\newcommand{\SupervisorTitle}{звание} %
%%
%%
%Руководитель, утверждающий задание
\newcommand{\Head}{В.Г.\,Пак} % И. О. Фамилия руководителя подразделения (руководителя ОП)
\newcommand{\HeadDegree}{Руководитель ОП}% Только должность:
%Руководитель %ОП 
%Заведующий % кафедрой
%Директор % Высшей школы
%Зам. директора
\newcommand{\HeadDep}{} % заменить на краткую аббревиатуру подразделения или оставить пустым, если утверждает руководитель ОП

%%% Руководитель, принимающий заявление
\newcommand{\HeadAp}{И.О.\,Фамилия} % И. О. Фамилия руководителя подразделения (руководителя ОП)
\newcommand{\HeadApDegree}{Должность руководителя}% Только должность:   
%Руководитель ОП 
%Заведующий кафедрой
%Директор Высшей школы
\newcommand{\HeadApDep}{O} % заменить на краткую аббревиатуру подразделения или оставить пустым, если утверждает руководитель ОП
%%% Консультант по нормоконтролю
\newcommand{\ConsultantNorm}{Ю.Д.\,Заковряшин} % И. О. Фамилия консультанта по нормоконтролю. ТОЛЬКО из числа ППС!
\newcommand{\ConsultantNormDegree}{Старший преподаватель ВШИСиСТ} %
%%% Первый консультант
\newcommand{\ConsultantExtraFull}{Заковряшин Юрий Дмитриевич} % Фамилия Имя Отчетство дополнительного консультанта
\newcommand{\ConsultantExtra}{И.О.\,Фамилия} % И. О. Фамилия дополнительного консультанта 
\newcommand{\ConsultantExtraDegree}{должность, степень} % 
\newcommand{\ConsultantExtraVin}{И.О.\,Фамилию} % И. О. Фамилия дополнительного консультанта в винительном падеже (Кого? что? Руководителя ...)
\newcommand{\ConsultantExtraDegreeVin}{должность, степень} %  в винительном падеже (Кого? что? Руководителя ...)
%%% Второй консультант
\newcommand{\ConsultantExtraTwoFull}{Фамилия Имя Отчетство} % Фамилия Имя Отчетство дополнительного консультанта
\newcommand{\ConsultantExtraTwo}{И.О.\,Фамилия} % И. О. Фамилия дополнительного консультанта
\newcommand{\ConsultantExtraTwoDegree}{должность, степень} %
\newcommand{\ConsultantExtraTwoVin}{И.О.\,Фамилию} % И. О. Фамилия дополнительного консультанта в винительном падеже (Кого? что? Руководителя ...)
\newcommand{\ConsultantExtraTwoDegreeVin}{должность, степень} %  в винительном падеже (Кого? что? Руководителя ...)
\newcommand{\Reviewer}{И.О.\,Фамилия} % И. О. Фамилия резензента. Обязателен только для магистров.
\newcommand{\ReviewerDegree}{должность, степень} % 
%%
%%
\renewcommand{\thesisTitle}{Исследование применения технологии обучения с подкреплением в управлении мультиагентной системой в игровой среде}
%\newcommand{\thesisDegree}{бакалавра}% магистра или специалиста%
\newcommand{\thesisDegree}{магистерская диссертация}% дипломный проект, дипломная работа, магистерская диссертация %c 2020
\newcommand{\thesisTitleEn}{Investigation of the Application of Reinforcement Learning to Managing Multi-Agent systems in Gaming Environment} %2020
\newcommand{\thesisDeadline}{05.2020}
\newcommand{\thesisStartDate}{03.02.2020}
\newcommand{\thesisYear}{2020}
%%
%%
\newcommand{\group}{в3540203/80277} % заменить вместо N номер группы
\newcommand{\thesisSpecialtyCode}{02.04.03}% код направления подготовки
\newcommand{\thesisSpecialtyTitle}{\mbox{Математическое} \mbox{обеспечение} и \mbox{администрирование} \mbox{информационных} \mbox{систем}} % наименование направления/специальности
\newcommand{\thesisOPPostfix}{02} % последние цифры кода образовательной программы (после <<_>>)
\newcommand{\thesisOPTitle}{\mbox{Проектирование} и \mbox{разработка} \mbox{информационных} \mbox{систем}}% наименование образовательной программы
%%
%%
\newcommand{\institute}{
%Название института
Институт компьютерных наук и технологий
%Гуманитарный институт
%Инженерно-строительный институт
%Институт биомедицинских систем и технологий
%Институт металлургии, машиностроения и транспорта
%Институт передовых производственных технологий
%Институт прикладной математики и механики
%Институт физики, нанотехнологий и телекоммуникаций
%Институт физической культуры, спорта и туризма
%Институт энергетики и транспортных систем
%Институт промышленного менеджмента, экономики и торговли
}%
%%
%%




%%% Задание ключевых слов и аннотации
%%
%%
%% Ключевых слов от 3 до 5 слов или словосочетаний в именительном падеже именительном падеже множественного числа (или в единственном числе, если нет другой формы) по правилам русского языка!!!
%%
%%
\newcommand{\keywordsRu}{Машинное обучение, обучение с подкреплением, глубокое обучение, мультиагентные системы} % ВВЕДИТЕ ключевые слова по-русски
%%
%%
\newcommand{\keywordsEn}{Machine Learning, Reinforcement Learning, Deep Learning, Multi-Agent Systems} % ВВЕДИТЕ ключевые слова по-английски
%%
%%
%% Реферат ОТ 1000 ДО 1500 знаков на русский или английский текст
%%
%Реферат должен содержать:
%- предмет, тему, цель ВКР;
%- метод или методологию проведения ВКР:
%- результаты ВКР:
%- область применения результатов ВКР;
%- выводы.

\newcommand{\abstractRu}{В данной работе исследована технология обучения с подкреплением, её применимость к мультиагентным системам, а именно применимость алгоритма мультиагентной глубокой детерминированной политики градиента (MADDPG). Изучено общение агентов между собой. Были предложены, реализованы и протестированы различные способы оптимизации обучения. Эксперименты производились в различных сценариях в двухмерной среде multiagent-particle-envs \cite{multiagent-particle-envs} от компании OpenAI.} % ВВЕДИТЕ текст аннотации по-русски
%%
%%
\newcommand{\abstractEn}{In this thesis Reinforcement Learning and its applicability to Multi-Agent systems were examined. Specifically, Multi-Agent Deep Deterministic Policy Gradient was reviewed. Communication between agents was researched. Different optimizing methods were suggested, implemented, and tested. Experiments were set up on different scenarios in 2D-environment  multiagent-particle-envs \cite{multiagent-particle-envs} by OpenAI.} % ВВЕДИТЕ текст аннотации по-английски


%%% РАЗДЕЛ ДЛЯ ОФОРМЛЕНИЯ ПРАКТИКИ
%Место прохождения практики
\newcommand{\PracticeType}{Отчет о прохождении %
	%стационарной производственной (технологической (проектно-технологической)) %
	такой-то % тип и вид ЗАМЕНИТЬ
	практики}

\newcommand{\Workplace}{СПбПУ, ИКНТ, ВШИСиСТ} % TODO Rename this variable

% Даты начала/окончания
\newcommand{\PracticeStartDate}{%
дд.мм.гггг%
%	22.06.2020
}%
\newcommand{\PracticeEndDate}{%
	дд.мм.гггг%
%	18.07.2020%
}%
%%

\newcommand{\School}{
%	Название высшей школы
	Высшая школа интеллектуальных систем и~суперкомпьютерных~технологий
}
\newcommand{\practiceTitle}{Тема практики}


%% ВНИМАНИЕ! Необходимо либо заменить текст аннотации (ключевых слов) на русском и английском, либо удалить там весь текст, иначе в свойства pdf-отчета по практике пойдет шаблонный текст.

%%% Не меняем дальнейшую часть - Do not modify the rest part
%%
%%
%%
%%
\ifnumequal{\value{docType}}{1}{% Если ВКР, то...
	\newcommand{\DocType}{Выпускная квалификационная работа}
	\newcommand{\pdfDocType}{\DocType~(\thesisDegree)} %задаём метаданные pdf файла
	\newcommand{\pdfTitle}{\thesisTitle}
}{% Иначе
	\newcommand{\DocType}{\PracticeType}
	\newcommand{\pdfDocType}{\DocType} %задаём метаданные pdf файла
	\newcommand{\pdfTitle}{\practiceTitle}
}%
\newcommand{\HeadTitle}{\HeadDegree~\HeadDep}
\newcommand{\HeadApTitle}{\HeadApDegree~\HeadApDep}
\newcommand{\thesisOPCode}{\thesisSpecialtyCode\_\thesisOPPostfix}% код образовательной программы
\newcommand{\thesisSpecialtyCodeAndTitle}{\thesisSpecialtyCode~\thesisSpecialtyTitle}% Код и наименование направления/специальности
\newcommand{\thesisOPCodeAndTitle}{\thesisOPCode~\thesisOPTitle} % код и наименование образовательной программы
%%
%%
\hypersetup{%часть болка hypesetup в style
		pdftitle={\pdfTitle},    % Заголовок pdf-файла
		pdfauthor={\AuthorFull},    % Автор
		pdfsubject={\pdfDocType. Шифр и наименование направления подготовки: \thesisSpecialtyCodeAndTitle. \abstractRu},      % Тема
		pdfcreator={LaTeX, SPbPU-student-thesis-template},     % Приложение-создатель
%		pdfproducer={},  % Производитель, Производитель PDF % будет выставлена автоматически
		pdfkeywords={\keywordsRu}
}
%%
%%
%% вспомогательные команды
\newcommand{\firef}[1]{рис.\ref{#1}} %figure reference
\newcommand{\taref}[1]{табл.\ref{#1}}	%table reference
%%
%%
%% Архивный вариант задания ключевых слов, аннотации и благодарностей 
% Too hard to export data from the environment to pdf-info
% https://tex.stackexchange.com/questions/184503/collecting-contents-of-environment-and-store-them-for-later-retrieval
%заменить NewEnviron на newenvironment для распознавания команды в TexStudio
%\NewEnviron{keywordsRu}{\noindent\MakeUppercase{\BODY}}
%\NewEnviron{keywordsEn}{\noindent\MakeUppercase{\BODY}}
%\newenvironment{abstractRu}{}{}
%\newenvironment{abstractEn}{}{}
%\newenvironment{acknowledgementsRu}{\par{\normalfont \acknowledgements.}}{}
%\newenvironment{acknowledgementsEn}{\par{\normalfont \acknowledgementsENG.}}{}


%%% Переопределение именований %%% Не меняем - Do not modify
%\newcommand{\Ministry}{Минобрнауки России} 
\newcommand{\Ministry}{Министерство науки и высшего образования Российской~Федерации} %с 2020
\newcommand{\SPbPU}{Санкт-Петербургский политехнический университет Петра~Великого}
\newcommand{\SPbPUOfficialPrefix}{Федеральное государственное автономное образовательное учреждение высшего образования}
\newcommand{\SPbPUOfficialShort}{ФГАОУ~ВО~<<СПбПУ>>}
%% Пробел между И. О. не допускается.
\renewcommand{\alsoname}{см. также}
\renewcommand{\seename}{см.}
\renewcommand{\headtoname}{вх.}
\renewcommand{\ccname}{исх.}
\renewcommand{\enclname}{вкл.}
\renewcommand{\pagename}{Pages}
\renewcommand{\partname}{Часть}
\renewcommand{\abstractname}{\textbf{Аннотация}}
\newcommand{\abstractnameENG}{\textbf{Annotation}}
\newcommand{\keywords}{\textbf{Ключевые слова}}
\newcommand{\keywordsENG}{\textbf{Keywords}}
\newcommand{\acknowledgements}{\textbf{Благодарности}}
\newcommand{\acknowledgementsENG}{\textbf{Acknowledgements}}
\renewcommand{\contentsname}{Content} % 
%\renewcommand{\contentsname}{Содержание} % (ГОСТ Р 7.0.11-2011, 4)
%\renewcommand{\contentsname}{Оглавление} % (ГОСТ Р 7.0.11-2011, 4)
\renewcommand{\figurename}{Рис.} % Стиль СПбПУ
%\renewcommand{\figurename}{Рисунок} % (ГОСТ Р 7.0.11-2011, 5.3.9)
\renewcommand{\tablename}{Таблица} % (ГОСТ Р 7.0.11-2011, 5.3.10)
%\renewcommand{\indexname}{Предметный указатель}
\renewcommand{\listfigurename}{Список рисунков}
\renewcommand{\listtablename}{Список таблиц}
\renewcommand{\refname}{\fullbibtitle}
\renewcommand{\bibname}{\fullbibtitle}

\newcommand{\chapterEnTitle}{Сhapter title} % <- input the English title here (only once!) 
\newcommand{\chapterRuTitle}{Название главы}          % <- введите 
\newcommand{\sectionEnTitle}{Section title} %<- input subparagraph title in english
\newcommand{\sectionRuTitle}{Название подраздела} % <- введите название подраздела по-русски
\newcommand{\subsectionEnTitle}{Subsection title} % - input subsection title in english
\newcommand{\subsectionRuTitle}{Название параграфа} % <- введите название параграфа по-русски
\newcommand{\subsubsectionEnTitle}{Subsubsection title} % <- input subparagraph title in english
\newcommand{\subsubsectionRuTitle}{Название подпараграфа} % <- введите название подпараграфа по-русски % Заполнить сведения, 
										 % в т.ч. ключевые слова и аннотацию.

%%% Титульник отчета по практике / Practice report title 
%%
%% добавить лист в pdf-навигацию 
%% add to pdf navigation menu
%%
\pdfbookmark[-1]{\pdfTitle}{tit}
%%
\thispagestyle{empty}%
\makeatletter
\newgeometry{top=2cm,bottom=2cm,left=3cm,right=1cm,headsep=0cm,footskip=0cm}
\savegeometry{NoFoot}%
\makeatother

%%% Распечатать версию документа / Print document version
%%
\begin{flushright}
	%	\vspace{0pt plus0.1fill}
	\boxed{\small
		\begin{tabular}{r} 
			\textbf{Пример отчета по практике <<SPbPU-student-thesis-template>>.} %\\ % перенос на новую строку
			\textbf{Версия от \today % \; время:  \currenttime. % время версии
			}
		\end{tabular}
	} %end boxed
	%	\vspace*{-5pt} % раскомментировать, если не хватает места
	\vspace{0pt plus0.1fill} % раскоментировать, если хватает места
\end{flushright}






% TODO Exact match of font size
%{\centering%
%	\footnotesize
%	\MakeUppercase{Федеральное государственное автономное образовательное учреждение\\высшего образования}\\%
%		{\bfseries <<\SPbPU>>\\%
%			\institute\\%
%			\School}
%}

{\centering%
	\small%
	\MakeUppercase{\SPbPUOfficialPrefix}\\
	{\bfseries %2020 - указание на изменения, которые могут быть введены в 2020 году
	<<\MakeUppercase{\SPbPU}>>\\%
	\MakeUppercase{\institute}\\
	\MakeUppercase{\School}
	}
\par}%
	

\vspace{0pt plus1fill} %число перед fill = кратность относительно некоторого расстояния fill, кусками которого заполнены пустые места


\noindent


%\vspace{0pt plus2fill} %


{\centering%
	{\bfseries{} 
	\DocType\\
	на тему: <<\practiceTitle>>}\\

\intervalS\normalfont%

	\uline{\AuthorFullVin , гр. \group}%

\intervalS\normalfont%

\par}%

%\intervalS% %ОБЯЗАТЕЛЬНО ДОБАВИТЬ ОТСТУП, ЕСЛИ ХВАТАЕТ МЕСТА


{\noindent {\bfseries Направление подготовки:} {\expandafter \ulined \thesisSpecialtyCode~\thesisSpecialtyTitle}.}\par


{\noindent {\bfseries Место прохождения практики:} {\expandafter \ulined \Workplace}.} % включая фактический адрес для практики в сторонней организации по договору


{\noindent {\bfseries Сроки практики:} \uline{с \PracticeStartDate~по \PracticeEndDate.}}\par


{\noindent {\bfseries Руководитель практики от \SPbPUOfficialShort:}} {\expandafter \ulined \SupervisorFull, \SupervisorJob, \SupervisorDegree.} %Ф.И.О., должность, степень



{\noindent \bfseries 
	Консультант практики от \SPbPUOfficialShort:
%	Консультанты практики от \SPbPUOfficialShort:
} 
{\noindent \expandafter \ulined \ConsultantExtraFull, \ConsultantExtraDegree}.%,      %% первый консультант Ф.И.О., должность, степень
%{\noindent \expandafter \ulined \ConsultantExtraTwoFull, \ConsultantExtraTwoDegree.}  %% второй консультант Ф.И.О., должность, степень 

{\noindent {\bfseries Оценка:} \uline{\hspace*{0.1\textheight}} % НЕ ЗАПОЛНЯЕМ!

\vspace{0pt plus1fill}%

\noindent
\begin{tabularx}{\linewidth}{lXl}
	Руководитель практики		&	&\\
	от \SPbPUOfficialShort		&	& \Supervisor     \\[\mfloatsep] % если не хватает места, закомментировать

	Консультант практики		&	&\\
	от \SPbPUOfficialShort		&	& \ConsultantExtra\\[\mfloatsep]
%							&	& \ConsultantExtraTwo\\[\mfloatsep]
	
	Обучающийся				&	&\Author\\[\mfloatsep]
	Дата: \uline{\PracticeEndDate}		&	&
\end{tabularx} %


%
\vspace{0pt plus4fill}% 

\restoregeometry
\newpage				 % Титульный лист
										 % Убираем footnotes, консультанта, если нет

\input{my_folder/contents}  	         % Оглавление


\chapter*{Введение} % * не проставляет номер
\addcontentsline{toc}{chapter}{Введение} % вносим в содержание

Данный пример выпускной квалификационной работы (далее~--- ВКР) создан для того, чтобы продемонстрировать возможности шаблонов SPbPU-student-templates, выполненных с помощью издательской системы \LaTeX{} \cite{spbpu-student-thesis-template}. В примере отображены некоторые обязательные элементы ВКР \cite{spbpu-student-thesis-specification}. Для того, чтобы подробнее ознакомиться с требованиями к наполнению этих элементов, а также с общими требованиями к структуре и оформлению ВКР, пожалуйста, ознакомьтесь с  \cite{spbpu-student-thesis-template-author-guide,spbpu-student-thesis-specification}.


Технология написания ВКР на \LaTeX{} подробно изложена в \cite{spbpu-student-thesis-template-author-guide}. В рекомендациях приведены ссылки на учебно-справочные материалы \LaTeX{} (под \LaTeX{} в документе может подразумеваться также \TeX, \LaTeXe).


Авторам, использующим \LaTeX{}, необходимо последовательно заменять текст данного шаблона в файлах <<\verb|thesis.tex|>> на текст своей ВКР, избегая при этом ошибок (errors) при компиляции. Синтаксические конструкции \LaTeX, которые задействованы в формировании того или иного текста выделены \texttt{машинописным шрифтом}. Иные шрифты в тексте ВКР (за исключением математических) использовать запрещено. 

Светлым курсивом выделены \textit{важные} элементы текста (ключевые слова определений, интонационные выделения словосочетаний), полужирным шрифтом --- \textbf{служебные} элементы текста (<<определение>>, <<теорема>>, <<лемма>> и т.п), а также при необходимости ключевые слова в алгоритмах. В соотношении к основному тексту курсив и полужирный шрифт не может превышать 1 \% текста на странице.
 
Полужирный курсив разрешено использовать только в \textbf{\textit{названиях подпараграфов (пунктов)}} и запрещёно использовать в основном тексте. 
\uline{Подчеркивание} допускается использовать только в задании в местах, где данные вписываются студентом, а также в математических формулах при необходимости.

Введение \textit{не должно превышать 4 страницы}. 


\textbf{Aктуальность исследования} заключается в N фактах и явлениях,  а также в их состоянии, связанных с ними нерешенных проблемах, слабо освещенных и требующих уточнения или дальнейшей разработки вопросов. 

\textbf{Объект исследования} --- это то, на что направлен процесс познания (индивид, коллектив, общность людей, сфера деятельности и т.п.). Связь объекта и предмета легко запоминаются по формуле: <<исследуем такой-то объект на предмет чего-то>>. Это процесс или явление, порождающее проблемную ситуацию, и избран-ное для изучения в целом. Всегда в объекте содержится предмет, а не наоборот. 

\textbf{Предмет исследования} --- один из аспектов, часть рассматриваемого объекта (свой-ства, состояния, процессы, направления и особенности деятельности структур по связям с общественностью, их сотрудников в конкретных сферах общественных отношений и т.д.). Предмет исследования частично совпадает с названием работы и содержится в цели сразу после сказуемого (<<выявить \ldots что?>>, <<определить \ldots что?>>, <<сформировать \ldots что?>>). Именно предмет исследования определяет тему выпускной квалификационной работы.
Объект и предмет исследования соотносятся между собой как целое и частное, общее и частности. 


\textit{Цель исследования} формулируется, исходя из проблемы, которую следует разрешить студенту в процессе выполнения выпускной квалификационной работы и представляет собой в самом сжатом виде тот результат (результаты), который должен быть получен в итоге исследования. Формулировку цели рекомендуется начинать со слов: <<сформировать/создать>>, <<разработать>>, <<провести>>, <<подготовить>>.

\textbf{Цель исследования} --- краткий ожидаемый результат, то есть решение практических задач и новые знания о рассматриваемом предмете исследования. 
В соответствии с целью исследования, логически определяются следующие \textbf{задачи работы} (должно быть \textit{не менее четырех задач, но не более шести задач}):

\begin{enumerate}
	\item Первая задача.
	\item Вторая задача.
	\item Третья задача.
	\item Четвертая задача.
\end{enumerate} 


Задачи отражают \textit{поэтапное достижение цели, при этом уточняют границы проводимого исследования}.
Рекомендуется формулировать задачи с глаголов в форме перечисления: <<изучить \ldots>>, <<выявить \ldots>>, <<проанализировать \ldots>>, <<разработать \ldots>>, <<описать \ldots>> и т.п. Заголовки выпускной квалификационной работы должны отражать суть поставленной задачи.


Общая направленность исследования задается до его начала сформулированными \textbf{гипотезами}, которыми могут быть:
\begin{itemize}
	\item научное предположение, выдвигаемое для объяснения каких-либо факторов, явлений и процессов, которые надо подтвердить или опровергнуть (т. е. требующее верификации);
	\item вероятностное знание, научно обоснованная догадка по объяснению действительности;
	\item прогноз ожидаемого решения проблемы, ответ на вопрос, поставленный в задаче;
	\item условно-категорическое умозаключение по схеме <<если \ldots, то \ldots>>, основными элементами которого являются условие (причина) и результат (следствие).
\end{itemize}  

Гипотеза --- это предполагаемое решение проблемы. В ходе исследования гипотезу проверяют и либо подтверждают, либо опровергают. Формулировка гипотезы \textit{обязательна только для магистров}.


\textbf{Теоретическая и методологическая база исследования}. В теоретической базе необходимо перечислить источники, которые использовались для написания работы. Приведём примеры ключевых фраз: 
\begin{itemize}
	\item <<Теоретической основой выпускной квалификационной работы послужили исследования  \ldots (перечисляются конкретные документы)>>.
	\item <<Практическая часть работы выполнялась на основании документов  \ldots>>.
	\item  <<При написании выпускной квалификационной работы использовалась работы отечественных и зарубежных специалистов \ldots>>.
	\item  <<Для выполнения анализа в практической части были использованы материалы  \ldots>>.
	\item <<При подготовке ВКР были использованы материалы таких учебных дисциплин, как "Технология конструкционных материалов'', "Экономика'' "Начертательная геометрия'' \ldots>>.
	\item <<При выполнении ВКР использовались материалы N организации \ldots (ссылка на официальный сайт)>>.
\end{itemize}

\textbf{Методологическая база исследования} должна содержать указание на методы и подходы, на которых основывается данная ВКР. 

Среди методов исследования студенту необходимо обратить внимание на общенаучные методы, включающие эмпирические (наблюдение, эксперимент, сравнение, описание, измерение), теоретические (формализация, аксиоматический, гипотетико-дедуктивный, восхождение от абстрактного к конкретному) и общелогические (анализ, абстрагирование, обобщение, идеализация, индукция, аналогия, моделирование и др.) методы.
Также следует назвать конкретно-научные (частные) методы научного познания, представляющие собой специфические методы конкретных наук: экономики, социологии, психологии, истории, логики и проч.

\textbf{Информационной базой} для разработки ВКР служат материалы, собранные студентом в процессе обучения в ВУЗе, в ходе прохождения учебной и производственной практик, а также во время прохождения преддипломной практики.
Дополнительная информационная база может включать информацию официальных статистических публикаций (например, Госкомстата России), материалы, получаемые из Интернета, информацию международных организаций и ассоциаций. 

\textbf{Степень научной разработанности проблемы} --- это состояние теоретической разработанности проблемы, анализ работ отечественных и зарубежных авторов, исследующих эту проблему. Здесь важно подчеркнуть исторические, экономические, политические или профессиональные явления, повлиявшие на выбор темы. Также в данной части введения проводится критический обзор современного состояния и освещения исследуемой темы в научной, профессиональной литературе и СМИ, обобщаются и оцениваются точки зрения различных авторов по теме исследования. 

\textbf{Научная новизна} выявляется в результате анализа литературных источников, уточнения концептуальных положений, обобщения опыта решения подобных проблем. Это принципиально новое знание, полученное в науке в ходе проведенного исследования (теоретические положения, впервые сформулированные и обоснованные, собственные методические рекомендации, которые можно использовать в практике).
Научная новизна выпускной квалификационной работы может состоять: 
\begin{itemize}
	\item в изучении фактов и явлений с помощью специальных научных методов и междисциплинарных подходов;
    \item в изучении уже известного в науке явления на новом экспериментальном материале;
	\item в переходе от качественного описания известных в науке фактов к их точно определяемой количественной характеристике;
	\item в изучении известных в науке явлений и процессов более совершенными методами;
	\item в сопоставлении, сравнительном анализе протекания процессов и явлений;
	\item в изменении условий протекания изучаемых процессов;
	\item в уточнении категориального аппарата дисциплины, определение типологии, признаков, специфики изучаемого явления.
\end{itemize}

\textbf{Практическая значимость} подробно отражается в:
\begin{itemize}
	\item практических рекомендациях или разработанном автором выпускной квалификационной работы проекте (как основная часть выпускной квалификационной работы);
	\item выявлении важности решения избранной проблемы для будущей деятельности магистра по выбранному направлению подготовки.
\end{itemize} 

Практическая значимость выпускной квалификационной работы может заключаться в возможности:

\begin{itemize}
	\item решения той или иной практической задачи в сфере профессиональной деятельности;
	\item проведения дальнейших научных исследований по теме ВКР;
	\item разработки конкретного проекта, направленного на интенсификацию работы исследуемой организации, предприятия.
\end{itemize}

\textbf{Апробация результатов} исследования включает:
\begin{itemize}
	\item участие в конференции, семинарах и т. д.;
	\item публикации по теме выпускной квалификационной работы;
	\item применение результатов исследования в практической области;
	\item разработку и внедрение конкретного проекта;
	\item выступления на научных конференциях, симпозиумах, форумах и т.п. (\textit{обязательно});
	\item публикации студента, включенные в список использованных источников. 
\end{itemize}


В силу ограниченности объема необходимо очень тщательно подойти к написанию введения, которое должно стать <<визитной карточкой>>, кратко, но емко характеризующей работу. Во введение не включают схемы, таблицы, описания, рекомендации и т.п. 

Целью первой главы, как правило, является всесторонний анализ предмета и объекта исследования, второй --- разработка предложений (алгоритмов, технологий и т.п.) по улучшению какого-либо процесса, протекающих с участием предмета и объекта исследования, третьей --- практическая реализация (имплементация) --- предложений (алгоритмов, технологий и т.п.) в виде программного (или иного) продукта, четвертой --- апробация разработанных в работе предложений и выводы целесообразности их дальнейшей разработки (использованию). 
Содержание глав в данном шаблоне приведено только для демонстрации возможностей \LaTeX.


%% Вспомогательные команды - Additional commands
%\newpage % принудительное начало с новой страницы, использовать только в конце раздела
%\clearpage % осуществляется пакетом <<placeins>> в пределах секций
%\newpage\leavevmode\thispagestyle{empty}\newpage % 100 % начало новой строки	    	 % Введение

%% Начало основной части
\chapter{Название первой главы: всестороннее изучение объекта и предмета исследования, анализ результатов, полученных другими авторами} \label{ch1}

\section{Введение. Сложносоставное название первого параграфа первой главы для~демонстрации переноса слов в содержании} \label{ch1:intro}

Хорошим стилем является наличие введения к главе. Во введении может быть описана цель написания главы, а также приведена краткая структура главы. Например, в параграфе \ref{ch1:sec1} приведены примеры оформления одиночных формул, рисунков и таблицы. Параграф \ref{ch1:sec2} посвящён многострочным формулам и сложносоставным рисункам.

Текст данной главы призван привести \textit{краткие} примеры оформления текстово-графических объектов. Более подробные примеры можно посмотреть в следующей главе, а также в рекомендациях студентам \cite{spbpu-student-thesis-template-author-guide}. 


\subsection{Название первого подпараграфа первого параграфа первой главы для~демонстрации переноса слов в содержании} % ~ нужен, чтобы избавиться от висячего предлога (союза) в конце строки

Содержание первого подпараграфа первого параграфа первой главы.
	
\section{Название параграфа} \label{ch1:sec1}


Одиночные формулы оформляют в окружении \texttt{equation}, например, как указано в следующей одиночной нумерованной формуле:
%
%
\begin{equation}% лучше не оставлять пропущенную строку (\par) перед окружениями для избежания лишних отсупов в pdf
\label{eq:Pi-ch1} % eq - equations, далее название, ch поставлено для избежания дублирования
\pi \approx 3,141.
\end{equation}
%
%
\begin{figure}[ht!] 
	\center
	\includegraphics [scale=0.27] {my_folder/images//spbpu_hydrotower}
	\caption{Вид на гидробашню СПбПУ \cite{spbpu-gallery}} 
	\label{fig:spbpu_hydrotower}  
\end{figure}
%
%
%\begin{table} [htbp]% Пример оформления таблицы
%	\centering\small
%	\caption{Представление данных для сквозного примера по ВКР \cite{Peskov2004}}%
%	\label{tab:ToyCompare}		
%		\begin{tabular}{|l|l|l|l|l|l|}
%			\hline
%			$G$&$m_1$&$m_2$&$m_3$&$m_4$&$K$\\
%			\hline
%			$g_1$&0&1&1&0&1\\ \hline
%			$g_2$&1&2&0&1&1\\ \hline
%			$g_3$&0&1&0&1&1\\ \hline
%			$g_4$&1&2&1&0&2\\ \hline
%			$g_5$&1&1&0&1&2\\ \hline
%			$g_6$&1&1&1&2&2\\ \hline		
%		\end{tabular}	
%	\normalsize% возвращаем шрифт к нормальному
%\end{table}


% \firef{} от figure reference
% \taref{} от table reference
% \eqref{} от equation reference

На \firef{fig:spbpu_hydrotower} изображена гидробашня СПбПУ, а в \taref{tab:ToyCompare} приведены данные, на примере которых коротко и наглядно будет изложена суть ВКР.


\section{Название параграфа} \label{ch1:sec2} 



Формулы могут быть размещены в несколько строк. Чтобы выставить номер формулы напротив средней строки, используйте окружение \verb|multlined| из пакета \verb|mathtools| следующим образом \cite{Ganter1999}:
%
\begin{equation} 
\label{eq:fConcept-order-ch1}
\begin{multlined}
(A_1,B_1)\leq (A_2,B_2)\; \Leftrightarrow \\  \Leftrightarrow\; A_1\subseteq A_2\; \Leftrightarrow \\ \Leftrightarrow\; B_2\subseteq B_1. 
\end{multlined}
\end{equation}


Используя команду \verb|\labelcref| из пакета \verb|cleveref|, допустимо следующим образом оформлять ссылку на несколько формул:
(\labelcref{eq:Pi-ch1,eq:fConcept-order-ch1}).
%
%
\input{my_folder/tex/fig-spbpu-whitehall-three-in-one} % пример подключения 3х иллюстрации в одном рисунке

Пример ссылок \cite{Article,Book,Booklet,Conference,Inbook,Incollection,Manual,Mastersthesis,Misc,Phdthesis,Proceedings,Techreport,Unpublished,badiou:briefings}, а также ссылок с указанием страниц, на котором отображены номера страниц  \cite[с.~96]{Naidenova2017} или в виде мультицитаты на несколько источников \cites[с.~96]{Naidenova2017}[с.~46]{Ganter1999}. Часть библиографических записей носит иллюстративный характер и не имеет отношения к реальной литературе. 



%\FloatBarrier % заставить рисунки и другие подвижные (float) элементы остановиться

\section{Выводы} \label{ch1:conclusion}

Текст выводов по главе \thechapter.


%% Вспомогательные команды - Additional commands
%
%\newpage % принудительное начало с новой страницы, использовать только в конце раздела
%\clearpage % осуществляется пакетом <<placeins>> в пределах секций
%\newpage\leavevmode\thispagestyle{empty}\newpage % 100 % начало новой страницы	         	 % Глава 1
\ContinueChapterBegin % размещать главы <<подряд>> 
\chapter{Название второй главы: разработка метода, алгоритма, модели исследования} \label{ch2}
	
\section{Введение} \label{ch2:intro}

Глава посвящена более подробным примерам оформления текстово-графических объектов.

В параграфе \ref{ch2:title-abbr} приведены примеры оформления многострочной формулы и одиночного рисунка. Параграф \ref{ch2:sec-abbr} раскрывает правила оформления перечислений и псевдокода. В параграфе \ref{ch2:sec-very-short-title} приведены примеры оформления сложносоставных рисунков, длинных таблиц, а также теоремоподобных окружений.


\section{Название параграфа} \label{ch2:title-abbr} %название по-русски



%%%%
%%		
%%  \input{...} commands are used only to sychronize some parts of the text with the author guide. Authors are free to type the text directly in .tex-files   
%%  \input{...} комманды используются только, чтобы синхронизировать части текта с рекомендациями авторам. Авторы  вольны вносить текст непосредственно в файл главы  
%%  
 \input{my_folder/tex/eq-Galois} % пример двух выравнивания двух формул в окружении align


На \firef{fig:spbpu-new-bld-autumn-ch2} приведёна фотография Нового научно-исследовательского корпуса СПбПУ.

	\begin{figure}[ht] 
	\center
	\includegraphics [scale=0.27] {my_folder/images/spbpu_new_bld_autumn}
	\caption{Новый научно-исследовательский корпус СПбПУ \cite{spbpu-gallery}} 
	\label{fig:spbpu-new-bld-autumn-ch2}  
	\end{figure}
	


	
\section{Название параграфа} \label{ch2:sec-abbr} %название по-русски
	
Название параграфа оформляется с помощью команды \verb|\section{...}|, название главы --- \verb|\chapter{...}|. 
	

\subsection{Название подпараграфа} \label{ch2:subsec-title-abbr} %название по-русски


Название параграфа оформляется с помощью команды  \texttt{\textbackslash{}subsection\{...\}}.
	
			
\subsubsection{Название подподпараграфа} \label{ch2:subsubsec-title-abbr} %название по-русски
	
	
Название подпараграфа оформляется с помощью команды  \texttt{\textbackslash{}subsubsecti\-on\{...\}}.



Перечисления могут быть с нумерационной частью и без неё и использоваться с иерархией и без иерархии. Нумерационная часть при этом формируется следующим способом:

\begin{enumerate}[1.]
	\item в перечислениях {\itshape без иерархии} оформляется арабскими цифрами с точкой (или длинным тире).
	\item В перечислениях {\itshape с иерархией} -- в последовательности сначала прописных латинских букв с точкой, затем арабских цифр с точкой и далее --- строчных латинских букв со скобкой.
\end{enumerate}


%% Если в дальнейшем нужно сделать сслыку на один из элементов нумеруемого перечисления, то нужно использовать конктрукцию типа:

%\begin{enumerate}[label=\arabic{enumi}.,ref=\arabic{enumi}]
%	\item text 1 \label{item:text1}
%	\item text 2
%\end{enumerate}
%\ref{item:text1}.


Далее приведён пример перечислений с иерархией.


\begin{enumerate}
	\item Первый пункт.
	\item Второй пункт.
	\item Третий пункт.
	\item По ГОСТ 2.105--95 \cite{gost-russian-text-documents} первый уровень нумерации идёт буквами русского или латинского алфавитов ({\itshape для определённости выбираем английский алфавит}),
	а второй "--- цифрами. 
	\begin{enumerate}
		\item В данном пункте лежит следующий нумерованный список: 
		\begin{enumerate}
			\item первый пункт;
			\item третий уровень нумерации не нормирован ГОСТ 2.105--95 ({\itshape для определённости выбираем английский алфавит});
			\item обращаем внимание на строчность букв в этом нумерованном и следующем маркированном списке:
			\begin{itemize}
				\item первый пункт маркированного списка.
			\end{itemize}    
		\end{enumerate}
	\end{enumerate}
	\item Пятый пункт верхнего уровня перечисления.
\end{enumerate}

Маркированный список (без нумерационной части) используется, если нет необходимости ссылки на определённое положение в списке:
\begin{itemize}
	\item первый пункт c {\itshape маленькой буквы} по правилам русского языка;
	\item второй пункт c {\itshape маленькой буквы} по правилам русского языка.
\end{itemize} % правила использования перечислений	

	
Оформление псевдокода необходимо осуществлять с помощью пакета \verb|algorithm2e| в окружении \verb|algorithm|. Данное окружение интерпретируется в шаблоне как рисунок. Пример оформления псевдокода алгоритма приведён на \firef{alg:AlgoFDSCALING}. 
	
	
\input{my_folder/tex/pseudocode-agl-DTestsFDScaling} % пример оформления псевдокода алгоритма 	

	
	\section{Название параграфа} \label{ch2:sec-very-short-title} %название по-русски


	
%% ВНИМАНИЕ: для того, чтобы избежать лишнего отступа между текстом  и формулами, пожалуйста, начинайте формулы без пропуска строки в исходном коде как в строках #2 и #3.
Одиночные формулы так же, как и отдельные формулы в составе группы, могут быть размещены в несколько строк. Чтобы выставить номер формулы напротив средней строки, используйте окружение \verb|multlined| из пакета \verb|mathtools| следующим образом \cite{Ganter1999}:
\begin{equation} % \tag{S} % tag - вписывает свой текст 
\label{eq:fConcept-order-G}
\begin{multlined}
(A_1,B_1)\leq (A_2,B_2)\; \Leftrightarrow \\  \Leftrightarrow\; A_1\subseteq A_2\; \Leftrightarrow \\ \Leftrightarrow\; B_2\subseteq B_1. 
\end{multlined}
\end{equation}

	
Используя команду \verb|\labelcref{...}| из пакета \verb|cleveref|, допустимо оформить ссылку на несколько формул, например, (\labelcref{eq:UpArrow-G,eq:DownArrow-G,eq:fConcept-order-G}). % пример оформления одиночной формулы в несколько строк

\input{my_folder/tex/fig-spbpu-sc-four-in-one} % пример подключения 4х иллюстраций в одном рисунке

%\input{my_folder/tex/fig-spbpu-whitehall-three-in-one} % пример подключения 3х иллюстрации в одном рисунке
%
%\input{my_folder/tex/fig-spbpu-main-bld-two-in-one} % пример подключения 2х иллюстраций в одном рисунке

Приведём пример табличного представления данных с записью продолжения на следующей странице на \taref{tab:long}.

%%% отладка longtable
%% 1) для контроля выхода таблицы за границы полей выставляем showframe в \geometry{}, см настройки
%% 2) используем \\* для запрета переноса определённой строки или средства из:
%% https://tex.stackexchange.com/q/344270/44348
%% 3) в крайнем случае для принудительного переноса таблицы на новую страницу используем \pagebreak после \\
\noindent % for correct centering
\begingroup
\centering
\small %выставляем шрифт в 12bp
\begin{longtable}[c]{|l|l|l|l|l|l|}
	\caption{Пример задания данных из \cite{Peskov2004} (с повтором для переноса таблицы на новую страницу)}%
	\label{tab:long}% label всегда желательно идти после caption
	\\
	\hline
	$G$&$m_1$&$m_2$&$m_3$&$m_4$&$K$\\ \hline
	1&2&3&4&5&6\\ \hline
	\endfirsthead%
	\captionsetup{format=tablenocaption,labelformat=continued} % до caption!
	\caption[]{}\\ % печать слов о продолжении таблицы
	\hline
	1&2&3&4&5&6\\ \hline
	\endhead
	\hline
	\endfoot
	\hline
	\endlastfoot
	$g_1$&0&1&1&0&1\\ \hline
	$g_2$&1&2&0&1&1\\ \hline
	$g_3$&0&1&0&1&1\\ \hline
	$g_4$&1&2&1&0&2\\ \hline
	$g_5$&1&1&0&1&2\\ \hline
	$g_6$&1&1&1&2&2\\ \hline
%
	$g_1$&0&1&1&0&1\\ \hline 
	$g_2$&1&2&0&1&1\\ \hline
	$g_3$&0&1&0&1&1\\ \hline
	$g_4$&1&2&1&0&2\\ \hline \noalign{\penalty-5000} % способствуем переносу на следующую стр
	$g_5$&1&1&0&1&2\\ \hline 
	$g_6$&1&1&1&2&2\\ \hline
%
	$g_1$&0&1&1&0&1\\ \hline 
	$g_2$&1&2&0&1&1\\ \hline
	$g_3$&0&1&0&1&1\\ \hline
	$g_4$&1&2&1&0&2\\ \hline
	$g_5$&1&1&0&1&2\\ \hline
	$g_6$&1&1&1&2&2\\ \hline
%		
	$g_1$&0&1&1&0&1\\ \hline 
	$g_2$&1&2&0&1&1\\ \hline
	$g_3$&0&1&0&1&1\\ \hline
	$g_4$&1&2&1&0&2\\ \hline
	$g_5$&1&1&0&1&2\\ \hline
	$g_6$&1&1&1&2&2\\ \hline
%
	$g_1$&0&1&1&0&1\\ \hline 
	$g_2$&1&2&0&1&1\\ \hline
	$g_3$&0&1&0&1&1\\ \hline
	$g_4$&1&2&1&0&2\\ \hline
	$g_5$&1&1&0&1&2\\ \hline
	$g_6$&1&1&1&2&2\\ \hline
%
	$g_1$&0&1&1&0&1\\ \hline 
	$g_2$&1&2&0&1&1\\ \hline
	$g_3$&0&1&0&1&1\\ \hline
	$g_4$&1&2&1&0&2\\ \hline
	$g_5$&1&1&0&1&2\\ \hline
	$g_6$&1&1&1&2&2\\ \hline
%
	$g_1$&0&1&1&0&1\\ \hline 
	$g_2$&1&2&0&1&1\\ \hline
	$g_3$&0&1&0&1&1\\ \hline
	$g_4$&1&2&1&0&2\\ \hline
	$g_5$&1&1&0&1&2\\ \hline
	$g_6$&1&1&1&2&2\\ \hline
\end{longtable}
\normalsize% возвращаем шрифт к нормальному
\endgroup % пример подключения таблицы на несколько страциц


\begin{table} [htbp]% Пример оформления таблицы
	\centering\small
	\caption{Пример представления данных для сквозного примера по ВКР \cite{Peskov2004}}%
	\label{tab:ToyCompare}		
		\begin{tabular}{|l|l|l|l|l|l|}
			\hline
			$G$&$m_1$&$m_2$&$m_3$&$m_4$&$K$\\
			\hline
			$g_1$&0&1&1&0&1\\ \hline
			$g_2$&1&2&0&1&1\\ \hline
			$g_3$&0&1&0&1&1\\ \hline
			$g_4$&1&2&1&0&2\\ \hline
			$g_5$&1&1&0&1&2\\ \hline
			$g_6$&1&1&1&2&2\\ \hline		
		\end{tabular}
%	\caption*{\raggedright\hspace*{2.5em} Составлено (или/и рассчитано) по \cite{Peskov2004}} %Если проведена авторская обработка или расчеты по какому-либо источнику	
	\normalsize% возвращаем шрифт к нормальному
\end{table}



%% please, before using, read the author guide carefully

\input{my_folder/tex/tab-toy-context-minipage} % пример подключения minipage

\input{my_folder/tex/fig-spbpu-new-bld-autumn-minipage} % пример подключения minipage




\input{my_folder/tex/rules-theorem-like-expressions} 

По аналогии с нумерацией формул, рисунков и таблиц нумеруются и иные текстово-графические объекты, то есть включаем в нумерацию номер главы, например: теорема 3.1. для первой теоремы третьей главы монографии. Команды \LaTeX{} выставляют нумерацию и форматирование автоматически. Полный перечень команд для подготовки текстово-графических и иных объектов находится в подробных методических рекомендациях \cite{spbpu-bci-template-author-guide}. 


\input{my_folder/tex/rules-list-of-environments} % список некоторых окружений


\input{my_folder/tex/theorem-example} %пример оформления теоремы


\input{my_folder/tex/definition-example} %пример оформления определёния


Вместо теоремо-подобных окружений для вставки небольших текстово-графических объектов иногда используются команды. Типичным примером такого подхода является команда \verb|\footnote{text}|\footnote{Внимание! Команда вставляется непосредственно после слова, куда вставляется сноска (без пробела). Лишние пробелы также не указываются внутри команды перед и после фигурных скобок.}, где в аргументе \verb|text| указывают текст \textit{подстрочной ссылки (сноски)}.В них \textit{нельзя добавлять веб-ссылки или цитировать литературу}. Для этих целей используется список литературы. Нумерация сносок сквозная по ВКР без точки на конце выставляется в шаблоне автоматически, однако в каждом приложении к ВКР нумерация, зависящая от номера приложения, выставляется префикс <<П>>, например <<П1.1>> --- первая сноска первого приложения. 




%\FloatBarrier % заставить рисунки и другие подвижные (float) элементы остановиться


\section{Выводы} \label{ch2:conclusion}

Текст заключения ко второй главе. Пример ссылок \cite{Article,Book,Booklet,Conference,Inbook,Incollection,Manual,Mastersthesis,Misc,Phdthesis,Proceedings,Techreport,Unpublished,badiou:briefings}, а также ссылок с указанием страниц, на котором отображены те или иные текстово-графические объекты  \cite[с.~96]{Naidenova2017} или в виде мультицитаты на несколько источников \cites[с.~96]{Naidenova2017}[с.~46]{Ganter1999}. Часть библиографических записей носит иллюстративный характер и не имеет отношения к реальной литературе. 

Короткое имя каждого библиографического источника содержится в специальном файле \verb|my_biblio.bib|, расположенном в папке \verb|my_folder|. Там же находятся исходные данные, которые с помощью программы \texttt{Biber} и стилевого файла \texttt{Biblatex-GOST} \cite{ctan-biblatex-gost} приведены в списке использованных источников согласно ГОСТ 7.0.5-2008.
Многообразные реальные примеры исходных библиографических данных можно посмотреть по ссылке \cite{ctan-biblatex-gost-examples}.

Как правило, ВКР должна состоять из четырех глав. Оставшиеся главы можно создать по образцу первых двух и подключить с помощью команды \verb|\input| к исходному коду ВКР. Далее в приложении \ref{appendix-MikTeX-TexStudio} приведены краткие инструкции запуска исходного кода ВКР \cite{latex-miktex,latex-texstudio}.

В приложении \ref{appendix-extra-examples} приведено подключение некоторых текстово-графических объектов. Они оформляются по приведенным ранее правилам. В качестве номера структурного элемента вместо номера главы используется <<П>> с номером главы. Текстово-графические объекты из приложений не учитываются в реферате.



%% Вспомогательные команды - Additional commands
%
%\newpage % принудительное начало с новой страницы, использовать только в конце раздела
%\clearpage % осуществляется пакетом <<placeins>> в пределах секций
%\newpage\leavevmode\thispagestyle{empty}\newpage % 100 % начало новой страницы	         	 % Глава 2
\input{my_folder/chapter3}           	 % Глава 3
\input{my_folder/chapter4}           	 % Глава 3
\ContinueChapterEnd % завершить размещение глав <<подряд>>
%% Завершение основной части

\input{my_folder/conclusion}        	 % Заключение

%% Наличие следующих перечней не исключает расшифровку сокращения и условного обозначения при первом упоминании в тексте!
\chapter*{Список сокращений и условных обозначений}             % Заголовок
\addcontentsline{toc}{chapter}{Список сокращений и условных обозначений}  % Добавляем его в оглавление
\noindent
\addtocounter{table}{-1}% Нужно откатить на единицу счетчик номеров таблиц, так как следующая таблица сделана для удобства представления информации по ГОСТ
%\begin{longtabu} to \dimexpr \textwidth-5\tabcolsep {r X}
\begin{longtabu} to \textwidth {r X} % Таблицу не прорисовываем!
% Жирное начертание для математических символов может иметь
% дополнительный смысл, поэтому они приводятся как в тексте
% диссертации
\textbf{МАС}  & Мультиагентная система. \label{acr:mas} \\
\textbf{ML}  & Машинное обучение (Machine Learning). \label{acr:ml} \\
\textbf{DL}  & Глубокое обучение (Deep Learning). \label{acr:dl} \\
\textbf{RL}  & Обучение с подкреплением (Reinforcement Learning). \label{acr:rl} \\
\textbf{DRL}  & Глубокое обучение с подкреплением (Deep Reinforcement Learning). \label{acr:drl} \\
\textbf{DPG}  & Детерминированный градиент политики (Deterministic Policy Gradient). \label{acr:dpg} \\
\textbf{DDPG}  & Глубокий детерминированный градиент политики (Deep Deterministic Policy Gradient). \label{acr:ddpg} \\
\textbf{MADDPG}  & Мультиагентный глубокий детерминированный градиент политики (Multiagent Deep Deterministic Policy Gradient). \label{acr:maddpg} \\
\textbf{COMA}  & Контрафактный мультиагентный градиент политики (Counterfactual Multi-Agent Policy Gradient, COMA). \label{acr:coma} \\
\textbf{ИНС}  & Искусственная нейронная сеть (Artificial Neural Network). \label{acr:ann} \\
\textbf{МППР}  & Марковский процесс принятия решений (Markov Decision Process, MDP). \label{acr:mdp} \\
%$\begin{rcases}
%a_n\\
%b_n
%\end{rcases}$  & 
%\begin{minipage}{\linewidth}
%Коэффициенты разложения Ми в дальнем поле, соответствующие
%электрическим и магнитным мультиполям.
%\end{minipage}
%\\
%${\boldsymbol{\hat{\mathrm e}}}$ & Единичный вектор. \\
%$E_0$ & Амплитуда падающего поля.\\
%$\begin{rcases}
%a_n\\
%b_n
%\end{rcases}$  & 
%Коэффициенты разложения Ми в дальнем поле соответствующие
%электрическим и магнитным мультиполям ещё раз, но без окружения
%minipage нет вертикального выравнивания по центру.
%\\
%$j$ & Тип функции Бесселя.\\
%$k$ & Волновой вектор падающей волны.\\
%
%$\begin{rcases}
%a_n\\
%b_n
%\end{rcases}$  & 
%\begin{minipage}{\linewidth}
%\vspace{0.7em}
%Коэффициенты разложения Ми в дальнем поле соответствующие
%электрическим и магнитным мультиполям, теперь окружение minipage есть
%и добавленно много текста, так что описание группы условных
%обозначений значительно превысило высоту этой группы... Для отбивки
%пришлось добавить дополнительные отступы.
%\vspace{0.5em}
%\end{minipage}
%\\
%$L$ & Общее число слоёв.\\
%$l$ & Номер слоя внутри стратифицированной сферы.\\
%$\lambda$ & Длина волны электромагнитного излучения
%в вакууме.\\
%$n$ & Порядок мультиполя.\\
%$\begin{rcases}
%{\mathbf{N}}_{e1n}^{(j)}&{\mathbf{N}}_{o1n}^{(j)}\\
%{\mathbf{M}_{o1n}^{(j)}}&{\mathbf{M}_{e1n}^{(j)}}
%\end{rcases}$  & Сферические векторные гармоники.\\
%$\mu$  & Магнитная проницаемость в вакууме.\\
%$r,\theta,\phi$ & Полярные координаты.\\
%$\omega$ & Частота падающей волны.\\
%
%  \textbf{BEM} & Boundary element method, метод граничных элементов.\\
%  \textbf{CST MWS} & Computer Simulation Technology Microwave Studio.
\end{longtabu}
		         % Необязательная рубрика! Список сокращений и условных обозначений

\input{my_folder/dictionary}    		 % Необязательная рубрика! Словарь терминов
% По порядку после Списка сокращений и условных обозначений, если есть.	


\input{my_folder/references}		     % Список литературы

% Здесь можно поместить список иллюстративного материала

\appendix % не редактировать / keep unmodified


\chapter{Краткие инструкции по настройке издательской системы \LaTeX}\label{appendix-MikTeX-TexStudio}							% Заголовок
%\addcontentsline{toc}{chapter}{Second call for chapters to participate in the book Machine learning in analysis of biomedical and socio-economic data}	% Добавляем его в оглавление

В SPbPU-BCI-template {\itshape автоматически выставляются необходимые настройки и в исходном тексте шаблона приведены примеры оформления текстово-графических объектов}, поэтому авторам достаточно заполнить имеющийся шаблон текстом главы (статьи), не вдаваясь в детали оформления, описанные далее. Возможный <<быстрый старт>> оформления главы (статьи) под Windows следующий\footnote{Внимание! Пример оформления подстрочной ссылки (сноски).}:

\begin{enumerate}
	\item Установка полной версии MikTeX  \cite{latex-miktex}.  В процессе установки лучше выставить параметр доустановки пакетов <<на лету>>.
	
	\item Установка TexStudio \cite{latex-texstudio}.
	
%		\item установка шрифтов PSCyr для работы с TimesNew\-Roman\-PSMT  	\href{https://github.com/AndreyAkinshin/Russian-Phd-LaTeX-Dissertation-Template/blob/master/PSCyr/Windows.md}{по данной инструкции}. В итоговом документе будет, скорее всего, использован Newton.
	
%	\item Переименование следующих файлов, где вместо \texttt{AuthorsSur\-names} необходимо подставить фамилии авторов (можно сокращать до первых четырех букв): 
%	
%	\begin{enumerate}
%		\item Основной файл \texttt{Book\_title\_ch\_Authors\-Sur\-names.tex}.
%		\item Библиография \texttt{biblio\textbackslash{}Book\_title\_bib\_Authors\-Sur\-na\-mes\-.bib}.
%		\item Пользовательские настройки (при необходимости), \texttt{common\textbackslash{}Book\_\-tit\-le\_ext\_Authors\-Sur\-names.tex}. 
%	\end{enumerate}
%	
%	\item После открытия основного файла \texttt{Book\_title\_ch\_Authors\-Sur\-names.tex} (с новым названием)   переименовать названия по аналогии в следующих командах \texttt{\textbackslash{}input\{\}}:
%	
%	\begin{enumerate}
%		\item \texttt{biblio/Book\_title\_bib\_Authors\-Sur\-names.bib},
%		\item \texttt{common/Book\_title\_ext\_Authors\-Sur\-names.tex (при необходимости) }.
%	\end{enumerate}
%	
	
	\item Запуск TexStudio и компиляция \verb|my_chapter.tex| с помощью команды <<Build\&View>> (например, с помощью двойной зелёной стрелки в верхней панели). {\itshape Иногда, для достижения нужного результата необходимо несколько раз скомпилировать документ.}
	
	\item В случае, если не отобразилась библиография, можно
	
	\begin{itemize}
		\item воспользоваться командой Tools $\to$ Commands $\to$ Biber, затем запустив Build\&View;
		
		\item настроить автоматическое включение библиографии в настройках Options $\to$ Configure TexStudio $\to$ Build $\to$  Build\&View (оставить по умолчанию, если сборка происходит слишком долго): \texttt{txs:///pdflatex | txs:///biber | txs:///pdflatex | txs:///pdflatex | txs:///\-view-pdf}.
	\end{itemize}
	
\end{enumerate}

В случае возникновения ошибок, попробуйте скомпилировать документ до последних действий или внимательно ознакомьтесь с описанием проблемы в log-файле. Бывает полезным переход (по подсказке TexStudio) в нужную строку в pdf-файле или запрос с текстом ошибке в поисковиках. Наиболее вероятной проблемой при первой компиляции может быть отсутствие какого-либо установленного пакета \LaTeX. 

В случае корректной работы настройки <<установка на лету>> все дополнительные пакеты будут скачиваться и устанавливаться в автоматическом режиме. Если доустановка пакетов осуществляется медленно (несколько пакетов за один запуск компилятора), то можно попробовать установить их в ручном режиме следующим образом:

\begin{enumerate}[1.]
	\item Запустите программу: меню $\to$ все программы $\to$ MikTeX $\to$ Maintenance (Admin) $\to$ MiKTeX Package Manager (Admin).
	\item Пользуясь поиском, убедитесь, что нужный пакет присутствует, но не установлен (если пакет отсутствует воспользуйтесь сначала MiKTeX Update (Admin)).
	\item Выделив строку с пакетом (возможно выбрать несколько или вообще все неустановленные пакеты), выполните установку Tools $\to$ Install или с помощью контекстного меню.
	\item После завершения установки запустите программу MiKTeX Settings (Admin).
	\item Обновите базу данных имен файлов Refresh FNDB.
\end{enumerate}


Для проверки текста статьи на русском языке полезно также воспользоваться настройками Options $\to$ Configure TexStudio $\to$ Language Checking $\to$  Default Language. Если русский язык <<ru\_RU>> не будет доступен в меню выбора, то необходимо вначале выполнить Import Dictionary, скачав из интернета любой русскоязычный словарь. 


%\chapter{\normalfont\normalsize{}Часто задаваемые вопросы (FAQ)}\label{Appendix-FAQ}							% Заголовок
%%\addcontentsline{toc}{chapter}{Second call for chapters to participate in the book Machine learning in analysis of biomedical and socio-economic data}	% Добавляем его в оглавление


Далее приведены формулы \eqref{eq:Pi-app2}, \eqref{eq:Pi-app2-},  \firef{fig:spbpu_hydrotower-app2}, \firef{fig:spbpu_hydrotower-app2-}, \taref{tab:ToyCompare-app2}, \taref{tab:ToyCompare-app2-}.


\begin{equation}% лучше не оставлять пропущенную строку (\par) перед окружениями для избежания лишних отсупов в pdf
\label{eq:Pi-app2-} % eq - equations, далее название, ch поставлено для избежания дублирования
\pi \approx 3,141.
\end{equation}

%
\begin{figure}[ht!] 
	\center
	\includegraphics [scale=0.27] {my_folder/images//spbpu_hydrotower}
	\caption{Вид на гидробашню СПбПУ \cite{spbpu-gallery}} 
	\label{fig:spbpu_hydrotower-app2-}  
\end{figure}

\begin{table} [htbp]% Пример оформления таблицы
	\centering\small
	\caption{Представление данных для сквозного примера по ВКР \cite{Peskov2004}}%
	\label{tab:ToyCompare-app2-}		
	\begin{tabular}{|l|l|l|l|l|l|}
		\hline
		$G$&$m_1$&$m_2$&$m_3$&$m_4$&$K$\\
		\hline
		$g_1$&0&1&1&0&1\\ \hline
		$g_2$&1&2&0&1&1\\ \hline
		$g_3$&0&1&0&1&1\\ \hline
		$g_4$&1&2&1&0&2\\ \hline
		$g_5$&1&1&0&1&2\\ \hline
		$g_6$&1&1&1&2&2\\ \hline		
	\end{tabular}	
	\normalsize% возвращаем шрифт к нормальному
\end{table}




\section{Параграф приложения}\label{app-2-1}							


\subsection{Название подпараграфа} \label{ch2:subsec-title-abbr} %название по-русски


Название параграфа оформляется с помощью команды  \texttt{\textbackslash{}subsection\{...\}}.


\subsubsection{Название подподпараграфа}\label{ch2:subsubsec-title-abbr} %название по-русски

\begin{equation}% лучше не оставлять пропущенную строку (\par) перед окружениями для избежания лишних отсупов в pdf
\label{eq:Pi-app2} % eq - equations, далее название, ch поставлено для избежания дублирования
\pi \approx 3,141.
\end{equation}
%
%
\begin{figure}[ht!] 
	\center
	\includegraphics [scale=0.27] {my_folder/images//spbpu_hydrotower}
	\caption{Вид на гидробашню СПбПУ \cite{spbpu-gallery}} 
	\label{fig:spbpu_hydrotower-app2}  
\end{figure}
%




\begin{table}[t!]% Пример оформления таблицы
	\centering\small
	\caption{Представление данных для сквозного примера по ВКР \cite{Peskov2004}}%
	\label{tab:ToyCompare-app2}		
	\begin{tabular}{|l|l|l|l|l|l|}
		\hline
		$G$&$m_1$&$m_2$&$m_3$&$m_4$&$K$\\
		\hline
		$g_1$&0&1&1&0&1\\ \hline
		$g_2$&1&2&0&1&1\\ \hline
		$g_3$&0&1&0&1&1\\ \hline
		$g_4$&1&2&1&0&2\\ \hline
		$g_5$&1&1&0&1&2\\ \hline
		$g_6$&1&1&1&2&2\\ \hline		
	\end{tabular}	
	\normalsize% возвращаем шрифт к нормальному
\end{table}


%% В случае, когда таблица (рисунок) размещаются на последней странице, для переноса названия приложения на новую строку используем:
\NewPage % начать новое приложение с новой страницы 			     % Приложение 1

\chapter{Некоторые дополнительные примеры}\label{appendix-extra-examples}							% 

В приложении\footnote{Внимание! Пример оформления подстрочной ссылки (сноски).} приведены формулы \eqref{eq:Pi-app}, \eqref{eq:Pi-app-}, \firef{fig:spbpu_hydrotower-app}, \firef{fig:spbpu_hydrotower-app-}, \taref{tab:ToyCompare-app}, \taref{tab:ToyCompare-app-}


\begin{equation}% лучше не оставлять пропущенную строку (\par) перед окружениями для избежания лишних отсупов в pdf
\label{eq:Pi-app-} % eq - equations, далее название, ch поставлено для избежания дублирования
\pi \approx 3,141.
\end{equation}
%
%
\begin{figure}[ht!] 
	\center
	\includegraphics [scale=0.27] {my_folder/images//spbpu_hydrotower}
	\caption{Вид на гидробашню СПбПУ \cite{spbpu-gallery}} 
	\label{fig:spbpu_hydrotower-app-}  
\end{figure}

\begin{table} [htbp]% Пример оформления таблицы
	\centering\small
	\caption{Представление данных для сквозного примера по ВКР \cite{Peskov2004}}%
	\label{tab:ToyCompare-app-}		
	\begin{tabular}{|l|l|l|l|l|l|}
		\hline
		$G$&$m_1$&$m_2$&$m_3$&$m_4$&$K$\\
		\hline
		$g_1$&0&1&1&0&1\\ \hline
		$g_2$&1&2&0&1&1\\ \hline
		$g_3$&0&1&0&1&1\\ \hline
		$g_4$&1&2&1&0&2\\ \hline
		$g_5$&1&1&0&1&2\\ \hline
		$g_6$&1&1&1&2&2\\ \hline		
	\end{tabular}	
	\normalsize% возвращаем шрифт к нормальному
\end{table}




\section{Подраздел приложения}\label{app-2-1}							


\begin{equation}% лучше не оставлять пропущенную строку (\par) перед окружениями для избежания лишних отсупов в pdf
\label{eq:Pi-app} % eq - equations, далее название, ch поставлено для избежания дублирования
\pi \approx 3,141.
\end{equation}
%
%
\begin{figure}[ht!] 
	\center
	\includegraphics [scale=0.27] {my_folder/images//spbpu_hydrotower}
	\caption{Вид на гидробашню СПбПУ \cite{spbpu-gallery}} 
	\label{fig:spbpu_hydrotower-app}  
\end{figure}

\begin{table} [htbp]% Пример оформления таблицы
	\centering\small
	\caption{Представление данных для сквозного примера по ВКР \cite{Peskov2004}}%
	\label{tab:ToyCompare-app}		
	\begin{tabular}{|l|l|l|l|l|l|}
		\hline
		$G$&$m_1$&$m_2$&$m_3$&$m_4$&$K$\\
		\hline
		$g_1$&0&1&1&0&1\\ \hline
		$g_2$&1&2&0&1&1\\ \hline
		$g_3$&0&1&0&1&1\\ \hline
		$g_4$&1&2&1&0&2\\ \hline
		$g_5$&1&1&0&1&2\\ \hline
		$g_6$&1&1&1&2&2\\ \hline		
	\end{tabular}	
	\normalsize% возвращаем шрифт к нормальному
\end{table}

			 	 % Приложение 2


\end{document} % конец документа


%%% Удачной защиты ВКР! - Good luck on the thesis defense!
%%
%%% Поддержать проект
%%
%% Запросы на добавление / изменение просим писать на следующей странице:
%% https://github.com/ParkhomenkoV/SPbPU-student-thesis-template/issues
%%
%% Список пожеланий в файле шаблона <<TO-DO-list.tex>>
%%
%% Благодарности просим указывать в виде 
%%
%% 1. Добавление <<Звезды>> проекту https://github.com/ParkhomenkoV/SPbPU-student-thesis-template/stargazers
%%
%% 2. Добавления <<Сердечка>> и репоста проекта в социальных сетях:
%%		https://vk.com/latex_polytech 
%%		https://www.fb.com/groups/latex.polytech
%%

%%% Support project
%%
%% Requests on adding / modifications is better to be publishen on the following web-page:
%% https://github.com/ParkhomenkoV/SPbPU-student-thesis-template/issues
%%
%% Wishlist is in the template's file called <<TO-DO-list.tex>>
%%
%% Acknowledgements are better to be done in the form of 
%%
%% 1. Adding <<Star>> to the project https://github.com/ParkhomenkoV/SPbPU-student-thesis-template/stargazers
%%
%% 2. Adding <<Likes>> and Project repost in the social networks:
%%		https://vk.com/latex_polytech 
%%		https://www.fb.com/groups/latex.polytech
%% 

% Check list при передаче ВКР:
% - Количество страниц в Задании 2. Если нет, то комментирование последней строки в my_task.tex
% - Зачистка всех вспомогательных файлов (Clear auxilary files) и компиляция ВКР не менее 3х раз