\chapter*{Список сокращений и условных обозначений}             % Заголовок
\addcontentsline{toc}{chapter}{Список сокращений и условных обозначений}  % Добавляем его в оглавление
\noindent
\addtocounter{table}{-1}% Нужно откатить на единицу счетчик номеров таблиц, так как следующая таблица сделана для удобства представления информации по ГОСТ
%\begin{longtabu} to \dimexpr \textwidth-5\tabcolsep {r X}
\begin{longtabu} to \textwidth {r X} % Таблицу не прорисовываем!
% Жирное начертание для математических символов может иметь
% дополнительный смысл, поэтому они приводятся как в тексте
% диссертации
\textbf{МАС}  & Мультиагентная система. \label{acr:mas} \\
\textbf{ML}  & Машинное обучение (Machine Learning). \label{acr:ml} \\
\textbf{DL}  & Глубокое обучение (Deep Learning). \label{acr:dl} \\
\textbf{RL}  & Обучение с подкреплением (Reinforcement Learning). \label{acr:rl} \\
\textbf{DRL}  & Глубокое обучение с подкреплением (Deep Reinforcement Learning). \label{acr:drl} \\
\textbf{DPG}  & Детерминированный градиент политики (Deterministic Policy Gradient). \label{acr:dpg} \\
\textbf{DDPG}  & Глубокий детерминированный градиент политики (Deep Deterministic Policy Gradient). \label{acr:ddpg} \\
\textbf{MADDPG}  & Мультиагентный глубокий детерминированный градиент политики (Multiagent Deep Deterministic Policy Gradient). \label{acr:maddpg} \\
\textbf{COMA}  & Контрафактный мультиагентный градиент политики (Counterfactual Multi-Agent Policy Gradient, COMA). \label{acr:coma} \\
\textbf{ИНС}  & Искусственная нейронная сеть (Artificial Neural Network). \label{acr:ann} \\
\textbf{МППР}  & Марковский процесс принятия решений (Markov Decision Process, MDP). \label{acr:mdp} \\
%$\begin{rcases}
%a_n\\
%b_n
%\end{rcases}$  & 
%\begin{minipage}{\linewidth}
%Коэффициенты разложения Ми в дальнем поле, соответствующие
%электрическим и магнитным мультиполям.
%\end{minipage}
%\\
%${\boldsymbol{\hat{\mathrm e}}}$ & Единичный вектор. \\
%$E_0$ & Амплитуда падающего поля.\\
%$\begin{rcases}
%a_n\\
%b_n
%\end{rcases}$  & 
%Коэффициенты разложения Ми в дальнем поле соответствующие
%электрическим и магнитным мультиполям ещё раз, но без окружения
%minipage нет вертикального выравнивания по центру.
%\\
%$j$ & Тип функции Бесселя.\\
%$k$ & Волновой вектор падающей волны.\\
%
%$\begin{rcases}
%a_n\\
%b_n
%\end{rcases}$  & 
%\begin{minipage}{\linewidth}
%\vspace{0.7em}
%Коэффициенты разложения Ми в дальнем поле соответствующие
%электрическим и магнитным мультиполям, теперь окружение minipage есть
%и добавленно много текста, так что описание группы условных
%обозначений значительно превысило высоту этой группы... Для отбивки
%пришлось добавить дополнительные отступы.
%\vspace{0.5em}
%\end{minipage}
%\\
%$L$ & Общее число слоёв.\\
%$l$ & Номер слоя внутри стратифицированной сферы.\\
%$\lambda$ & Длина волны электромагнитного излучения
%в вакууме.\\
%$n$ & Порядок мультиполя.\\
%$\begin{rcases}
%{\mathbf{N}}_{e1n}^{(j)}&{\mathbf{N}}_{o1n}^{(j)}\\
%{\mathbf{M}_{o1n}^{(j)}}&{\mathbf{M}_{e1n}^{(j)}}
%\end{rcases}$  & Сферические векторные гармоники.\\
%$\mu$  & Магнитная проницаемость в вакууме.\\
%$r,\theta,\phi$ & Полярные координаты.\\
%$\omega$ & Частота падающей волны.\\
%
%  \textbf{BEM} & Boundary element method, метод граничных элементов.\\
%  \textbf{CST MWS} & Computer Simulation Technology Microwave Studio.
\end{longtabu}
