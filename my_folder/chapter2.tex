\chapter{Название второй главы: разработка метода, алгоритма, модели исследования} \label{ch2}
	
\section{Введение} \label{ch2:intro}

Глава посвящена более подробным примерам оформления текстово-графических объектов.

В параграфе \ref{ch2:title-abbr} приведены примеры оформления многострочной формулы и одиночного рисунка. Параграф \ref{ch2:sec-abbr} раскрывает правила оформления перечислений и псевдокода. В параграфе \ref{ch2:sec-very-short-title} приведены примеры оформления сложносоставных рисунков, длинных таблиц, а также теоремоподобных окружений.


\section{Название параграфа} \label{ch2:title-abbr} %название по-русски



%%%%
%%		
%%  \input{...} commands are used only to sychronize some parts of the text with the author guide. Authors are free to type the text directly in .tex-files   
%%  \input{...} комманды используются только, чтобы синхронизировать части текта с рекомендациями авторам. Авторы  вольны вносить текст непосредственно в файл главы  
%%  
 \input{my_folder/tex/eq-Galois} % пример двух выравнивания двух формул в окружении align


На \firef{fig:spbpu-new-bld-autumn-ch2} приведёна фотография Нового научно-исследовательского корпуса СПбПУ.

	\begin{figure}[ht] 
	\center
	\includegraphics [scale=0.27] {my_folder/images/spbpu_new_bld_autumn}
	\caption{Новый научно-исследовательский корпус СПбПУ \cite{spbpu-gallery}} 
	\label{fig:spbpu-new-bld-autumn-ch2}  
	\end{figure}
	


	
\section{Название параграфа} \label{ch2:sec-abbr} %название по-русски
	
Название параграфа оформляется с помощью команды \verb|\section{...}|, название главы --- \verb|\chapter{...}|. 
	

\subsection{Название подпараграфа} \label{ch2:subsec-title-abbr} %название по-русски


Название параграфа оформляется с помощью команды  \texttt{\textbackslash{}subsection\{...\}}.
	
			
\subsubsection{Название подподпараграфа} \label{ch2:subsubsec-title-abbr} %название по-русски
	
	
Название подпараграфа оформляется с помощью команды  \texttt{\textbackslash{}subsubsecti\-on\{...\}}.



Перечисления могут быть с нумерационной частью и без неё и использоваться с иерархией и без иерархии. Нумерационная часть при этом формируется следующим способом:

\begin{enumerate}[1.]
	\item в перечислениях {\itshape без иерархии} оформляется арабскими цифрами с точкой (или длинным тире).
	\item В перечислениях {\itshape с иерархией} -- в последовательности сначала прописных латинских букв с точкой, затем арабских цифр с точкой и далее --- строчных латинских букв со скобкой.
\end{enumerate}


%% Если в дальнейшем нужно сделать сслыку на один из элементов нумеруемого перечисления, то нужно использовать конктрукцию типа:

%\begin{enumerate}[label=\arabic{enumi}.,ref=\arabic{enumi}]
%	\item text 1 \label{item:text1}
%	\item text 2
%\end{enumerate}
%\ref{item:text1}.


Далее приведён пример перечислений с иерархией.


\begin{enumerate}
	\item Первый пункт.
	\item Второй пункт.
	\item Третий пункт.
	\item По ГОСТ 2.105--95 \cite{gost-russian-text-documents} первый уровень нумерации идёт буквами русского или латинского алфавитов ({\itshape для определённости выбираем английский алфавит}),
	а второй "--- цифрами. 
	\begin{enumerate}
		\item В данном пункте лежит следующий нумерованный список: 
		\begin{enumerate}
			\item первый пункт;
			\item третий уровень нумерации не нормирован ГОСТ 2.105--95 ({\itshape для определённости выбираем английский алфавит});
			\item обращаем внимание на строчность букв в этом нумерованном и следующем маркированном списке:
			\begin{itemize}
				\item первый пункт маркированного списка.
			\end{itemize}    
		\end{enumerate}
	\end{enumerate}
	\item Пятый пункт верхнего уровня перечисления.
\end{enumerate}

Маркированный список (без нумерационной части) используется, если нет необходимости ссылки на определённое положение в списке:
\begin{itemize}
	\item первый пункт c {\itshape маленькой буквы} по правилам русского языка;
	\item второй пункт c {\itshape маленькой буквы} по правилам русского языка.
\end{itemize} % правила использования перечислений	

	
Оформление псевдокода необходимо осуществлять с помощью пакета \verb|algorithm2e| в окружении \verb|algorithm|. Данное окружение интерпретируется в шаблоне как рисунок. Пример оформления псевдокода алгоритма приведён на \firef{alg:AlgoFDSCALING}. 
	
	
\input{my_folder/tex/pseudocode-agl-DTestsFDScaling} % пример оформления псевдокода алгоритма 	

	
	\section{Название параграфа} \label{ch2:sec-very-short-title} %название по-русски


	
%% ВНИМАНИЕ: для того, чтобы избежать лишнего отступа между текстом  и формулами, пожалуйста, начинайте формулы без пропуска строки в исходном коде как в строках #2 и #3.
Одиночные формулы так же, как и отдельные формулы в составе группы, могут быть размещены в несколько строк. Чтобы выставить номер формулы напротив средней строки, используйте окружение \verb|multlined| из пакета \verb|mathtools| следующим образом \cite{Ganter1999}:
\begin{equation} % \tag{S} % tag - вписывает свой текст 
\label{eq:fConcept-order-G}
\begin{multlined}
(A_1,B_1)\leq (A_2,B_2)\; \Leftrightarrow \\  \Leftrightarrow\; A_1\subseteq A_2\; \Leftrightarrow \\ \Leftrightarrow\; B_2\subseteq B_1. 
\end{multlined}
\end{equation}

	
Используя команду \verb|\labelcref{...}| из пакета \verb|cleveref|, допустимо оформить ссылку на несколько формул, например, (\labelcref{eq:UpArrow-G,eq:DownArrow-G,eq:fConcept-order-G}). % пример оформления одиночной формулы в несколько строк

\input{my_folder/tex/fig-spbpu-sc-four-in-one} % пример подключения 4х иллюстраций в одном рисунке

%\input{my_folder/tex/fig-spbpu-whitehall-three-in-one} % пример подключения 3х иллюстрации в одном рисунке
%
%\input{my_folder/tex/fig-spbpu-main-bld-two-in-one} % пример подключения 2х иллюстраций в одном рисунке

Приведём пример табличного представления данных с записью продолжения на следующей странице на \taref{tab:long}.

%%% отладка longtable
%% 1) для контроля выхода таблицы за границы полей выставляем showframe в \geometry{}, см настройки
%% 2) используем \\* для запрета переноса определённой строки или средства из:
%% https://tex.stackexchange.com/q/344270/44348
%% 3) в крайнем случае для принудительного переноса таблицы на новую страницу используем \pagebreak после \\
\noindent % for correct centering
\begingroup
\centering
\small %выставляем шрифт в 12bp
\begin{longtable}[c]{|l|l|l|l|l|l|}
	\caption{Пример задания данных из \cite{Peskov2004} (с повтором для переноса таблицы на новую страницу)}%
	\label{tab:long}% label всегда желательно идти после caption
	\\
	\hline
	$G$&$m_1$&$m_2$&$m_3$&$m_4$&$K$\\ \hline
	1&2&3&4&5&6\\ \hline
	\endfirsthead%
	\captionsetup{format=tablenocaption,labelformat=continued} % до caption!
	\caption[]{}\\ % печать слов о продолжении таблицы
	\hline
	1&2&3&4&5&6\\ \hline
	\endhead
	\hline
	\endfoot
	\hline
	\endlastfoot
	$g_1$&0&1&1&0&1\\ \hline
	$g_2$&1&2&0&1&1\\ \hline
	$g_3$&0&1&0&1&1\\ \hline
	$g_4$&1&2&1&0&2\\ \hline
	$g_5$&1&1&0&1&2\\ \hline
	$g_6$&1&1&1&2&2\\ \hline
%
	$g_1$&0&1&1&0&1\\ \hline 
	$g_2$&1&2&0&1&1\\ \hline
	$g_3$&0&1&0&1&1\\ \hline
	$g_4$&1&2&1&0&2\\ \hline \noalign{\penalty-5000} % способствуем переносу на следующую стр
	$g_5$&1&1&0&1&2\\ \hline 
	$g_6$&1&1&1&2&2\\ \hline
%
	$g_1$&0&1&1&0&1\\ \hline 
	$g_2$&1&2&0&1&1\\ \hline
	$g_3$&0&1&0&1&1\\ \hline
	$g_4$&1&2&1&0&2\\ \hline
	$g_5$&1&1&0&1&2\\ \hline
	$g_6$&1&1&1&2&2\\ \hline
%		
	$g_1$&0&1&1&0&1\\ \hline 
	$g_2$&1&2&0&1&1\\ \hline
	$g_3$&0&1&0&1&1\\ \hline
	$g_4$&1&2&1&0&2\\ \hline
	$g_5$&1&1&0&1&2\\ \hline
	$g_6$&1&1&1&2&2\\ \hline
%
	$g_1$&0&1&1&0&1\\ \hline 
	$g_2$&1&2&0&1&1\\ \hline
	$g_3$&0&1&0&1&1\\ \hline
	$g_4$&1&2&1&0&2\\ \hline
	$g_5$&1&1&0&1&2\\ \hline
	$g_6$&1&1&1&2&2\\ \hline
%
	$g_1$&0&1&1&0&1\\ \hline 
	$g_2$&1&2&0&1&1\\ \hline
	$g_3$&0&1&0&1&1\\ \hline
	$g_4$&1&2&1&0&2\\ \hline
	$g_5$&1&1&0&1&2\\ \hline
	$g_6$&1&1&1&2&2\\ \hline
%
	$g_1$&0&1&1&0&1\\ \hline 
	$g_2$&1&2&0&1&1\\ \hline
	$g_3$&0&1&0&1&1\\ \hline
	$g_4$&1&2&1&0&2\\ \hline
	$g_5$&1&1&0&1&2\\ \hline
	$g_6$&1&1&1&2&2\\ \hline
\end{longtable}
\normalsize% возвращаем шрифт к нормальному
\endgroup % пример подключения таблицы на несколько страциц


\begin{table} [htbp]% Пример оформления таблицы
	\centering\small
	\caption{Пример представления данных для сквозного примера по ВКР \cite{Peskov2004}}%
	\label{tab:ToyCompare}		
		\begin{tabular}{|l|l|l|l|l|l|}
			\hline
			$G$&$m_1$&$m_2$&$m_3$&$m_4$&$K$\\
			\hline
			$g_1$&0&1&1&0&1\\ \hline
			$g_2$&1&2&0&1&1\\ \hline
			$g_3$&0&1&0&1&1\\ \hline
			$g_4$&1&2&1&0&2\\ \hline
			$g_5$&1&1&0&1&2\\ \hline
			$g_6$&1&1&1&2&2\\ \hline		
		\end{tabular}
%	\caption*{\raggedright\hspace*{2.5em} Составлено (или/и рассчитано) по \cite{Peskov2004}} %Если проведена авторская обработка или расчеты по какому-либо источнику	
	\normalsize% возвращаем шрифт к нормальному
\end{table}



%% please, before using, read the author guide carefully

\input{my_folder/tex/tab-toy-context-minipage} % пример подключения minipage

\input{my_folder/tex/fig-spbpu-new-bld-autumn-minipage} % пример подключения minipage




\input{my_folder/tex/rules-theorem-like-expressions} 

По аналогии с нумерацией формул, рисунков и таблиц нумеруются и иные текстово-графические объекты, то есть включаем в нумерацию номер главы, например: теорема 3.1. для первой теоремы третьей главы монографии. Команды \LaTeX{} выставляют нумерацию и форматирование автоматически. Полный перечень команд для подготовки текстово-графических и иных объектов находится в подробных методических рекомендациях \cite{spbpu-bci-template-author-guide}. 


\input{my_folder/tex/rules-list-of-environments} % список некоторых окружений


\input{my_folder/tex/theorem-example} %пример оформления теоремы


\input{my_folder/tex/definition-example} %пример оформления определёния


Вместо теоремо-подобных окружений для вставки небольших текстово-графических объектов иногда используются команды. Типичным примером такого подхода является команда \verb|\footnote{text}|\footnote{Внимание! Команда вставляется непосредственно после слова, куда вставляется сноска (без пробела). Лишние пробелы также не указываются внутри команды перед и после фигурных скобок.}, где в аргументе \verb|text| указывают текст \textit{подстрочной ссылки (сноски)}.В них \textit{нельзя добавлять веб-ссылки или цитировать литературу}. Для этих целей используется список литературы. Нумерация сносок сквозная по ВКР без точки на конце выставляется в шаблоне автоматически, однако в каждом приложении к ВКР нумерация, зависящая от номера приложения, выставляется префикс <<П>>, например <<П1.1>> --- первая сноска первого приложения. 




%\FloatBarrier % заставить рисунки и другие подвижные (float) элементы остановиться


\section{Выводы} \label{ch2:conclusion}

Текст заключения ко второй главе. Пример ссылок \cite{Article,Book,Booklet,Conference,Inbook,Incollection,Manual,Mastersthesis,Misc,Phdthesis,Proceedings,Techreport,Unpublished,badiou:briefings}, а также ссылок с указанием страниц, на котором отображены те или иные текстово-графические объекты  \cite[с.~96]{Naidenova2017} или в виде мультицитаты на несколько источников \cites[с.~96]{Naidenova2017}[с.~46]{Ganter1999}. Часть библиографических записей носит иллюстративный характер и не имеет отношения к реальной литературе. 

Короткое имя каждого библиографического источника содержится в специальном файле \verb|my_biblio.bib|, расположенном в папке \verb|my_folder|. Там же находятся исходные данные, которые с помощью программы \texttt{Biber} и стилевого файла \texttt{Biblatex-GOST} \cite{ctan-biblatex-gost} приведены в списке использованных источников согласно ГОСТ 7.0.5-2008.
Многообразные реальные примеры исходных библиографических данных можно посмотреть по ссылке \cite{ctan-biblatex-gost-examples}.

Как правило, ВКР должна состоять из четырех глав. Оставшиеся главы можно создать по образцу первых двух и подключить с помощью команды \verb|\input| к исходному коду ВКР. Далее в приложении \ref{appendix-MikTeX-TexStudio} приведены краткие инструкции запуска исходного кода ВКР \cite{latex-miktex,latex-texstudio}.

В приложении \ref{appendix-extra-examples} приведено подключение некоторых текстово-графических объектов. Они оформляются по приведенным ранее правилам. В качестве номера структурного элемента вместо номера главы используется <<П>> с номером главы. Текстово-графические объекты из приложений не учитываются в реферате.



%% Вспомогательные команды - Additional commands
%
%\newpage % принудительное начало с новой страницы, использовать только в конце раздела
%\clearpage % осуществляется пакетом <<placeins>> в пределах секций
%\newpage\leavevmode\thispagestyle{empty}\newpage % 100 % начало новой страницы