\section{Прикладные методы}

\subsection{Архитектура ветвления действий (Action Branching)}

В сценариях, подразумевающих совместную работу нескольких агентов, наряду с физическими действиями важно иметь и коммуникационные действия. В некоторых случаях количество действий может быть большим, как в \cite{tavakoli2017action}. В игровых сценариях этой работы агент имеет два вида действия, которые представляют физическое движение и действие, выбранное для общения.

Полное действие в таких сценариях включает в себя два относительно независимых действия. Сеть \textit{акторов} проектируется таким образом, что она выделяет скрытые представления из наблюдений и имеет две «головы» на выходе. Одна возвращает физическое действие, вторая – действие для общения.

В соответствии с теорией архитектуры ветвления действий \cite{tavakoli2017action}, можно оптимизировать каждое измерение действия относительно независимо. Полное действие в итоге представляет собой объединение двух действий.

\subsection{Исследовательский шум (Exploration Noise)}

\textit{Исследование} и \textit{эксплуатация} (Exploration and exploitation) — это дилемма в обучении с подкреплением. \textit{Эксплуатация} — это следование агента текущей политике, с целью совершения "жадных" действий, которые приносят наибольшую награду. \textit{Исследование} берёт на себя риски, чтобы попробовать другие действия, которые потенциально могут принести лучшую награду в долгосрочной перспективе. \textit{Исследование} необходимо агенту, ищущему оптимальную политику, хотя оно кажется неоптимальным в нынешней ситуации и даёт меньшее вознаграждение. В обучении с подкреплением агент обычно больше \textit{исследует} окружающую среду в начале обучения. В ходе оптимизации политики с течением времени агент постепенно уменьшает и стабилизирует \textit{исследование} до низкого уровня и в итоге больше придерживается получившейся оптимальной политики.

Чтобы включить \textit{исследование}, на действия накладываются шумы. Какой шум применять, определяется настройкой среды. \textit{Процесс Орнштейна-Уленбека} используется в DDPG в [2] для получения коррелированных по времени \textit{исследований}. Этот процесс считается, эффективным для проблем физического контроля с инерцией. Гауссовское распределение шума используется для физических движений в [4]. В этой работе для наложения шума на действия, генерируемые сетью \textit{акторов}, выбирается стандартное \textit{гауссовское распределение}:

\begin{equation}
    \begin{multlined}
        \mu'_i(s_i) = \mu_i(s_i|\theta^{\mu_i}) + \mathcal{N}
    \end{multlined}
\end{equation},

где $\mu'$ - политика исследования, а $\mathcal{N}$ - шум исследования, который можно выбрать в зависимости от настроек среды, $\mathcal{N}$ затухает на каждом шаге со скоростью $\epsilon$, то есть $\mathcal{N} \leftarrow \epsilon \mathcal{N}$.
