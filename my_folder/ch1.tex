\chapter{Обзор технологии обучения с подкреплением} \label{ch1}

%\section{Введение} \label{ch1:intro}

Алгоритмы и методы глубокого обучения с подкреплением (Deep Reinforcement Learning, DRL) – это методы, которые используются для исследования вопросов данной работы. \hyperref[acr:drl]{DRL} – это подраздел машинного обучения, который лежит на пересечении обучения с подкреплением (Reinforcement Learning) и глубокого обучения (Deep Learning).

Прежде чем говорить о DRL, нужно кратко рассмотреть понятия, связанные с искусственным интеллектом и машинным обучением.

На \firef{fig:ch1-ML-and-AI-concepts} представлен обзор концепций, о которых идёт речь в этой главе. 

Сначала будут рассмотрены искусственный интеллект, машинное обучение и его три категории. Затем будет описано глубокое обучение. Наконец, будут рассмотрены основные алгоритмы глубокого обучения с подкреплением.

Таким образом можно будет понять, почему и как глубокое обучение и обучение с подкреплением объединяются в глубокое обучение с подкреплением, которое позволяет агентам взаимодействовать с более сложной средой и вести себя более разумно.

\begin{figure}[ht!] 
	\center
	\includegraphics [scale=0.80] {my_folder/images/ch1/ML-and-AI-concepts.png}
	\caption{Диаграмма, показывающая концепты машинного обучения и искуственного интеллекта и их отношения} 
	\label{fig:ch1-ML-and-AI-concepts}  
\end{figure}

\section{Искусственный интеллект} \label{ch1:ai}

Искусственный интеллект \cite{Crevier93} (\hyperref[acr:ai]{ИИ}), как следует из названия, является противопоставлением естественному интеллекту человека и животных в силу своего искусственного происхождения. Именно такой тип интеллекта люди реализуют в машинах.

Конечной целью ИИ является создание таких автономных систем, которые способны учиться методом многочисленных проб и ошибок, чтобы найти оптимальное поведение для достижения максимально возможных целей в сложившемся окружении. \cite{RussellAndNorvig-AI-modern-approach}

С 21 века, благодаря нескольким прорывам, ИИ доминирует в викторинах и настольных играх, превосходя уровень игры людей \cite{Watson} \cite{AlphaGo}. Наряду с увеличением вычислительной мощности, улучшением алгоритмов и доступностью больших наборов данных, ИИ развивался революционными темпами. В ближайшем будущем ИИ ещё сильнее повлияет на работу и повседневную жизнь человека.

\section{Машинное Обучение} \label{ch1:ml}

Если ИИ – это более широкая концепция машинного интеллекта для выполнения задач, которые пока выполняют люди, то Машинное обучение (ML) - это основной метод разработки ИИ, который не требует явного программирования \cite{Samuel-SomeStudies}. ML позволяет компьютерам строить модели и применять алгоритмы при изучение больших объемов данных. Модели ML обучаются с использованием методов статистики, чтобы понять структуру набора данных или последовательности экспериментов. Обученные модели могут распознавать паттерны и делать очень точные прогнозы учитывая не очевидные данные или обрабатывать определённые задачи в неочевидных сценариях \cite{bishop06pattern}.

Основные категории алгоритмов машинного обучения показаны на рисунке \firef{fig:ch1-ML-categories}. А именно обучение с учителем, обучение без учителя и обучение с подкреплением.
Категории определяются тем, как алгоритмы и модели наполняются данными и как данные анализируются.

\begin{figure}[ht!] 
	\center
	\includegraphics [scale=0.60] {my_folder/images/ch1/ML-categories.png}
	\caption{Категории ML и соответстующие алгоритмы. Основано на \cite{Sultan_2018}} 
	\label{fig:ch1-ML-categories}
\end{figure}
% TODO: перевести

\subsection{Обучение с учителем (Supervised Learning)}

\textbf{TODO: может быть добавить рисунок}

Один из способов машинного обучения, в ходе которого испытуемая система принудительно обучается с помощью примеров «стимул-реакция». С точки зрения кибернетики, является одним из видов кибернетического эксперимента. Между входами и эталонными выходами (стимул-реакция) может существовать некоторая зависимость, но она неизвестна. Известна только конечная совокупность прецедентов — пар «стимул-реакция», называемая обучающей выборкой. На основе этих данных требуется восстановить зависимость (построить модель отношений стимул-реакция, пригодных для прогнозирования), то есть построить алгоритм, способный для любого объекта выдать достаточно точный ответ \cite{james2014introduction}.

Обучение с учителем можно разделить на \textit{регрессию} и \textit{классификацию}, в зависимости от того, являются ли выходные переменные количественными или качественный. Количественные переменные принимают числовые значения, в то время как качественные переменные принимают значения в одном из K различных классов или категории \cite{james2014introduction}. Например, прогноз цены на жилье по данным параметрам, таким как местоположение дома, общая площадь и количество комнат и т. д., это проблема регрессии. Диагноз рака - проблема классификации поскольку выходной сигнал либо положительный, либо отрицательный.

В обучении с учителем набор данных делится на обучающий, набор для валидации и тестовый набор. Обучающий набор представляет собой пары ввода и вывода переменных, которые напрямую вводятся в модель для обучения. Валидационный набор используется для контроля за переобучением модели. Наконец, тестовый набор используется для подтверждения того, что обученная модель обобщена и точна

Обучение с учителем - наиболее распространенная категория в машинном обучении, но оно требует больших наборов данных с правильными «ответами». Это может быть очень дорого, в некоторых случаях не практично.


\subsection{Обучение без учителя (Unsupervised Learning)}
\textbf{TODO: может быть добавить рисунок}

Это ещё одна важная категория в машинном обучении.
В отличие от обучения с учителем, алгоритмы этой категории используют наборы данных, не размеченные «ответами». Задача в этой категории --- обнаружить скрытую структуру в данных и распределить их по группам.

Эта категория алгоритмов получила своё название из-за отсутствие меток или выходных переменных. Кластеризация является типичным инструментом, который используется чтобы понять связь между наблюдениями и распределить их в разные группы \cite{hastie2001elements}

При обучении без учителя данные не делятся на обучающие, проверочные и тестовые. Набор данных подается в модель напрямую и группируется в отдельные группы.


\subsection{Обучение с подкреплением (Reinforcement Learning)}

Последняя категория машинного обучения является междисциплинарной областью, которая сочетает в себе машинное обучение, неврологию, поведенческую психология, теорию управления и т. д. Цель обучения с подкреплением заключается в достижении целей без четких инструкций, но с наградами или штрафами, получаемыми от взаимодействий с окружающей средой. 

Агент обучения с подкреплением изучает оптимальную политику, последовательность действий, которая максимизируют общую будущую награду (reward) \cite{SuttonAndBarto-RL-Introduction-p2}.

В обучении с подкреплением агент наблюдает состояние ${s_t}$ на этапе времени ${t}$, затем он взаимодействует с окружающей средой, выполняя действие ${a_t}$. Среда переходит в следующее состояние ${s_{t+1}}$, учитывая текущее состояние и выбранное действие, которое ведёт к получению агентом вознаграждения ${r_t}$. Цель агента узнать политику $\pi$, которая сопоставляет состояния с действиями, так что последовательность действий, выбранная агентом, максимизирует ожидаемое будущее вознаграждение. На каждом шаге взаимодействия со средой агент генерирует переход ${\{s_t, a_t, s_{t+1}, r_t\}}$, который даёт информацию, необходимую для улучшения политики, см. \firef{fig:ch1-RL-flow}.

\begin{figure}[ht!] 
	\center
	\includegraphics [scale=0.60] {my_folder/images/ch1/rl-flow.png}
	\caption{Агент взаимодействует с окружающей средой, сначала наблюдая состояние, затем совершая действие и, наконец, получая вознаграждения за выбранное действия. Через многочисленные попытки и ошибки, агент учится формулировать оптимальную политику \cite{SuttonAndBarto-RL-Introduction-p50}} 
	\label{fig:ch1-RL-flow}
\end{figure}

\section{Глубокое обучение} \label{ch1:dl}

Глубокое обучение (Deep Learning, \hyperref[acr:dl]{DL}) - подраздел машинного обучения, в настоящее время достигает выдающихся результатов во многих областях, в том числе распознавание изображений, компьютерное зрение, распознавание речи и естественное языковая обработка. Глубокое обучение, как бы имитирует биологический мозг, обрабатывает информацию с помощью искусственных нейронных сетей.

Идея симуляции работы нейронов мозга человека зародилась десятилетия назад. Тем не менее, прорыв произошёл в последние годы, когда стали доступны большие наборы данных и вычислительные мощности \cite{10.1145/2771283}. Теперь можно строить модели с большим количеством слоёв искусственных нейронов, чем когда-либо прежде. С сильным увеличением глубины, сети достигают исключительной производительности в области распознавания изображений и речи. Считается, что глубокое обучение является одним из наиболее перспективных подходов для решения текущих задач ИИ.

\subsection{Искусственная нейронная сеть}

Мозг человека и животных - чрезвычайно сложный орган, который до сих пор до конца не изучен. Тем не менее, некоторые аспекты его структуры и функции были расшифрованы. Фундаментальным рабочим элементом в мозге является нейрон. Многочисленные нейроны связаны между собой сложным образом, что даёт возможность, запоминания, мышления и принятия решений.

Хотя нейроны сложны и функционируют разными способами, все они имеют некоторые базовые компоненты, такие как: ядро, дендриты, аксон и синапсы. Дендриты действуют как входные каналы, через которые нейроны получают информацию от синапсов других нейронов. Затем ядро обрабатывает эти сигналы и превращает в вывод, который затем отправляется в другие нейроны. Связь этих компонентов играет роль линии передачи в нейронных сетях.

Хоть \hyperref[acr:ann]{ИНС} и не так сложны, как человеческий мозг, они имитируют базовую структуру, которая включает входные, выходные слои, а также, обычно, скрытые слои. В каждом слое есть искусственные нейроны. Нейроны в одном слое обычно связаны с каждым нейроном в следующем слое. Они передают числовые сигналы через соединения с другими нейронами, примерно, как это происходит в биологической нейронной сети \cite{296402}.

Поскольку выходные сигналы нейронов ИНС представляются в виде вещественных чисел, выход обычно сравнивается с порогом. Только если порог превышен, нейрон передаёт сигнал следующим подключённым нейронам.

\begin{figure}[ht!]
	\center
	\includegraphics [scale=0.60] {my_folder/images/ch1/ANN.png}
	\caption{Базовая структура ИНС. ИНС обычно состоит из входного и выходного слоёв, а также пары скрытых слоёв. Основано на \cite{Khajanchi2003ArtificialNN} \cite{mitchell1997machine}}
	\label{fig:ch1-ANN}
\end{figure}

Связи между нейронами параметризованы весами, которые обновляются во время обучения, чтобы отрегулировать мощность сигналов, проходящих через сеть [24]. \firef{fig:ch1-ANN} показывает базовую структуру ИНС.


\subsection{Глубокая нейронная сеть}

Как уже упоминалось выше, концепция нейронной сети не нова.

На ранних этапах из-за ограничений вычислительной мощности, нейронные сети имели очень малую глубину. Обычно они содержали только входной, выходной слои и пару скрытых слоёв. Кроме того, количество нейронов в каждом слое так же было ограничено. Глубокие нейронные сети не находили практического применения до последних лет, когда стали доступны огромные вычислительные мощности и большие объёмы данных.

\begin{figure}[ht!]
	\center
	\includegraphics [scale=0.65] {my_folder/images/ch1/DRL-flow.png}
	\caption{Cравнение DRL и традиционной RL, где необходимо явно указать, какие действия производить в каких состояниях. Использование глубокой нейронной сети в DRL позволяет обучаться и принимать решения на основе необработанного сенсорного ввода}
	\label{fig:DRL-flow}
\end{figure}

Глубокие нейронные сети с несколькими слоями хороши для извлечения скрытых свойств \cite{7344858}. Каждый слой, решает свою собственную задачу. Узлы в каждом слое учатся на конкретных наборах свойств, которые приходят из предыдущего слоя. Теоретически чем глубже нейронная сеть, тем более сложны и абстрактны особенности, которые она может распознать, поскольку каждый нейрон агрегирует выводы из нейронов предыдущего слоя \cite{bengio2012representation}. Например, в задаче распознавания изображений вход представляет собой матрицу пикселей. Первый слой извлекает начальные объекты, такие как рёбра из пикселей, затем следующий слой кодирует расположение краёв и следующий слой распознаёт глаза, рот, уши, ноги, крылья, хвост и т. д. Наконец, последний слой распознаёт на изображении кошку, собаку или птицу. Таким образом, глубокие нейронные сети не нуждаются во вмешательстве людей, а сами изучают иерархию объектов.

С математической точки зрения нейронная сеть определяет функцию ${y = f(x; \theta)}$. Она описывает соответствие входных данных ${x}$ выходным ${y}$, где ${y}$ -- это категория в задаче классификации или выходное значение в задачах регрессии. Тренировка модели с использованием набора данных ведёт к вычислению параметра {$\theta$}.

После завершения обучения предполагается, что нейронная сеть аппроксимирует целевую функцию ${f^*}$ \cite{Goodfellow-et-al-2016}. Правильно обученная ИНС может лучше соответствовать набору данных, а также делает прогноз, учитывая неочевидные зависимости от данных.

Глубокие нейронные сети могут быть реализованы по-разному в зависимости от конкретных практических задач. Например, свёрточные нейронные сети специализируются на компьютерном зрении, а рекуррентные нейронных сети лучше подходят для обработки естественного языка.

Функцию отображения ${f(x)}$ можно рассматривать как цепочку из многих связанных функции в виде  ${f(x) = f^{(n)}(f^{(n-1)}(...f^{(2)}(f^{(1)}(x))))}$. ${n}$ связанных функций соответствуют глубине ИНС. Функция стоимости определяется на основе сравнения вывода ${y}$ из ${f(x)}$ с целевым значением ${t}$ из тренировочного набора данных. Цель обучения нейронной сети в том, чтобы приблизить функцию ${f(x)}$ к такой, чтобы минимизировать функцию стоимости. Минимизация функции стоимости является задачей оптимизации, и для этого часто используют алгоритм {\itshape градиентного спуска}. В {\itshape обратном распространении} (backpropagation), градиент используется для итеративного обновления нейронной сети во время обучения. Оптимизатор решает, какие параметры и как должны быть обновлены, а {\itshape скорость обучения} (learning rate) задаёт размер шага, на который параметр обновляется на каждой итерации.

\section{Обучение с подкреплением} \label{ch1:ml}

Хотя алгоритмы \hyperref[acr:rl]{RL} эффективно решают различные задачи, им не хватает масштабируемости и размерности.

С ростом глубоких нейронных сетей в последние годы, RL начинает использовать их функции приближения и представления свойств [14]. Это помогает преодолеть недостатки алгоритмов RL.

Это устраняет необходимость описывать свойства вручную, позволяя обучать модели, способные непосредственно выводить оптимальные действия, на основе необработанного и высокоразмерного ввода с сенсоров.

Таким образом, использование ИНС в обучении подкреплением, создает новую область – глубокое обучение с подкреплением (Deep Reinforcement Learning, DRL).

\subsection{Алгоритмы глубокого обучения с подкреплением}

\subsection{Подход, основанный на функции состояния (Value Based подход)}

\subsection{Линия поведения (Policy Based)}

\subsubsection{Policy Gradients}

Policy Gradients содержимое
\textbf{TODO: научиться писать формулы}
\begin{equation} 
	\label{eq:fConcept-order-ch1}
	\begin{multlined}
		(A_1,B_1)\leq (A_2,B_2)\; \Leftrightarrow \\  \Leftrightarrow\; A_1\subseteq A_2\; \Leftrightarrow \\ \Leftrightarrow\; B_2\subseteq B_1. 
	\end{multlined}
\end{equation}

\section{Выводы} \label{ch1:conclusion}

В этой главе были рассмотрены основные понятия, связанные с машинным обучением и обучением с подкреплением, в частности. Это позволяет перейти к рассмотрению мультиагентных систем и алгоритмов для работы с ними.

%% Вспомогательные команды - Additional commands
%
\newpage % принудительное начало с новой страницы, использовать только в конце раздела
%\clearpage % осуществляется пакетом <<placeins>> в пределах секций
%\newpage\leavevmode\thispagestyle{empty}\newpage % 100 % начало новой страницы