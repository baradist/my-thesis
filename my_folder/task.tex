%%%% Начало оформления заголовка - оставить без изменений !!! %%%%
\input{my_folder/task_settings}	% настройки - начало 
	
				{%\normalfont %2020
						\MakeUppercase{\SPbPU}}\\
				\institute

\par}\intervalS% завершает input

				\noindent
				\begin{minipage}{\linewidth}
				\vspace{\mfloatsep} % интервал 	
				\begin{tabularx}{\linewidth}{Xl}
					&УТВЕРЖДАЮ      \\
					&\HeadTitle     \\			
					&\underline{\hspace*{0.1\textheight}} \Head     \\
					&<<\underline{\hspace*{0.05\textheight}}>> \underline{\hspace*{0.1\textheight}} \thesisYear г.  \\  
				\end{tabularx}
				\vspace{\mfloatsep} % интервал 	
				\end{minipage}

\intervalS{\centering\bfseries%

				ЗАДАНИЕ\\
				на выполнение %с 2020 года 
				%по выполнению % до 2020 года
				выпускной квалификационной работы


\intervalS\normalfont%

				студенту \uline{\AuthorFullDat{} гр.~\group}


\par}\intervalS%
%%%%
%%%% Конец оформления заголовка  %%%%
 	
	
	
\begin{enumerate}[1.]
	\item Тема работы: {\expandafter \ulined \thesisTitle.}
	%\item Тема работы (на английском языке): \uline{\thesisTitleEn.} % вероятно после 2021 года
	\item Срок сдачи студентом законченной работы%\footnote{Определяется руководителем ОП, но не позднее последнего числа преддипломной практики и/или не позднее, чем за 20 дней до защиты в силу п. 6.1. <<Порядка обеспечения самостоятельности выполнения письменных работ и проверки письменных работ на объем заимствований>>.}: \uline{\thesisDeadline.} 
	\item Исходные данные по работе%\footnote{Текст, который подчёркнут и/или выделен в отдельные элементы нумерационного списка, приведён в качестве примера.}:
	\begin{enumerate}[label=\theenumi\arabic*.]
		\item Фреймворки для разработки алгоритмов обучения с подкреплением (pytorch и другие).
		\item Программные среды тестирования алгоритмов (OpenAI Gym и другие).
		\item Алгоритмы обучения с подкреплением.
		\item Теория управления МАС.
	\end{enumerate}
	% \printbibliographyTask % печать списка источников % КОММЕНТИРУЕМ ЕСЛИ НЕ ИСПОЛЬЗУЕТСЯ
	% В СЛУЧАЕ, ЕСЛИ НЕ ИСПОЛЬЗУЕТСЯ МОЖНО ТАКЖЕ ЗАЙТИ В setup.tex и закомментировать \vspace{-0.28\curtextsize}
	\item Содержание работы (перечень подлежащих разработке вопросов):
	\begin{enumerate}[label=\theenumi\arabic*.]
		\item Обзор технологии обучения с подкреплением.
		\item Обзор подходов к управлению мультиагентными системами (МАС).
		\item Обзор аналогичных исследований.
		\item Постановка задачи выпускной квалификационной работы.
		\item Разработка алгоритмов управления агентами в МАС. Их исследование.
		\item Внедрение разработанных алгоритмов в компьютерную игру, тестирование.
		\item Экспериментальное исследование алгоритмов.
		\item Заключение, выводы об эффективности исследованных алгоритмов.
	\end{enumerate}
	\item Перечень графического материала (с указанием обязательных чертежей): 
	\begin{enumerate}[label=\theenumi\arabic*.]
		\item Графики результатов экспериментов.
		\item Блок-схемы алгоритмов.
		\item Скриншоты
	\end{enumerate}	
		\item Консультанты по работе%\footnote{Подпись консультанта по нормоконтролю пока не требуется. Назначается всем по умолчанию.}:
		\begin{enumerate}[label=\theenumi\arabic*.] 
		% \item  \uline{\emakefirstuc{\ConsultantExtraDegree}, \ConsultantExtra.} % закомментировать при необходимости, идёт первый по порядку.
		\item
		{\emakefirstuc{\ConsultantNormDegree}, \ConsultantNorm{} (нормоконтроль).} %	Обязателен для всех студентов
	\end{enumerate}
		\item Дата выдачи задания%\footnote{Не позднее 3 месяцев до защиты (утверждение тем ВКР по университету) или первого числа преддипломной практики или по решению руководителя ОП или подразделения (открытый вопрос).}: \uline{\thesisStartDate.}
\end{enumerate}

\intervalS%можно удалить пробел

Руководитель ВКР \uline{\hspace*{0.1\textheight} \Supervisor}


\intervalS%можно удалить пробел

% Консультант\footnote{В случае, если есть консультант, отличный от консультанта по нормоконтролю.}  \uline{\hspace*{0.1\textheight}\ConsultantExtra}


\intervalS%можно удалить пробел

%Консультант по нормоконтролю \uline{\hspace*{0.1\textheight} \ConsultantNorm}%ПОКА НЕ ТРЕБУЕТСЯ, Т.К. ОН У ВСЕХ ПО УМОЛЧАНИЮ

Задание принял к исполнению \uline{\thesisStartDate}

\intervalS%можно удалить пробел

Студент \uline{\hspace*{0.1\textheight}  \Author}



\setcounter{tskPageLast}{\value{page}} %сохранили номер последней страницы Задания
\setcounter{tskPages}{\value{tskPageLast}-\value{tskPageFirst}}
\newrefsection % начинаем новую секцию библиографии
\newrefcontext % удаляем префикс к пунктам списка литературы
\restoregeometry % восстанавливаем настройки страницы
\pagestyle{plain} % удаляем номер страницы на первой/второй странице Задания
\setlength{\parindent}{2.5em} % восстанавливаем абзацный отступ
%% Обязательно закомментировать, если получается один лист в задании:
\ifnumequal{\value{tskPages}}{0}{% Если 1 страница в Задании, то ничего не делать.
}{% Иначе 
% до 2020 года требовалось печатать задание на 1 листе с двух сторон и не подсчитывать вторую страницу
%\setcounter{page}{\value{page}-\value{tskPages}} 	% вычесть значение tskPages при печати более 1 страницы страниц
}%
\AtNextBibliography{\setcounter{citenum}{0}}%обнуляем счетчик библиографии	% настройки - конец