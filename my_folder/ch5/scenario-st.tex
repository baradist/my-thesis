\subsection{Сценарий 4}

Как уже упоминалось \hyperref[exp-st]{выше}, этот сценарий не подразумевает коммуникационных действий. Но он был выбран нами, так как он достаточно сложный, и при этом агенты из одной команды имеют одинаковые пространства наблюдения и действий, что позволяет применить вариант алгоритма MADDPG с общим мозгом.

Мы поставили эксперимент с двумя преследователями и одной жертвой.

\begin{figure}[ht!]
	\center
%	\includegraphics [scale=0.6] {my_folder/images/ch5/st-rew.png}
	\includegraphics [scale=0.6] {my_folder/images/in_progress.png}
	\caption{Среднее вознаграждение преследователей и жертвы с алгоритмами DDPG, MADDPG и MADDPG с общим мозгом.}
	\label{fig:result-st-rew}
\end{figure}

\begin{figure}[ht!]
	\center
%	\includegraphics [scale=0.6] {my_folder/images/ch5/st-comm.png}
	\includegraphics [scale=0.6] {my_folder/images/in_progress.png}
	\caption{График консистентности действий коммуникации для алгоритмов DDPG, MADDPG и MADDPG с общим мозгом.}
	\label{fig:result-st-comm}
\end{figure}
